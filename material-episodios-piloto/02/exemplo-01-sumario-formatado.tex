\documentclass[12pt]{memoir}

\usepackage[utf8]{inputenc}
\usepackage[T1]{fontenc}
\usepackage[brazil]{babel}
\usepackage{tikz}
% FORMATAÇÃO DO SUMÁRIO 
%%%%%%%%%%%%%%%%%%%%%%%%%%%%%%%%%%%%%%

% Redefinir estilo do título do sumário
\renewcommand\printtoctitle[1]{\HUGE#1}

% redefinitions for chapter entries
\renewcommand\chapternumberline[1]{}

% nome do capítulo escrito grande
\renewcommand\cftchapterfont{\LARGE} 
% número da página escrito grande e em itálico
\renewcommand\cftchapterpagefont{\LARGE \itshape} 


% Formatação de cada linha
% do sumário correspondendo
% a um capítulo
% %%%%%%%%%%%%%%%%%%%%%%%%%%%%%%%%%%
% Estamos definindo aqui o comando \l@mychap
% 
\makeatletter
\newcommand\l@mychap[3]{%
  \bigskip% pulo vertical
  \par\noindent% novo parágrafo, sem indentação
  \parbox{\dimexpr\textwidth-15pt\relax}{%
    \cftchapterfont#1, {\cftchapterpagefont\textcolor{red}{#2}}%
  }\par%
}

% Este é o comando que o memoir usa, escrevemos ele
% de modo que use o comando \l@mychap acima.
\renewcommand*\l@chapter[2]{%
  \l@mychap{#1}{#2}{\chaptername}%
}
\makeatother
% %%%%%%%%%%%%%%%%%%%%%%%%%%%%%%%%%%%%%%%%%%%%%%%%%%%%%%%%%%%%%%%%%%%%
%%% fim da formatação do sumário
\begin{document}

\title{O Romance da Ciência}
\author{Carl Sagan}
\date {}
\maketitle
\newpage

\tableofcontents*

\chapter{O Cérebro de Broca}
\setcounter{page}{17}
\chapter{Podemos conhecer o Universo?}
\setcounter{page}{27}
\chapter{Aquele mundo que acena como uma libertação}
\setcounter{page}{33}
\chapter{Louvor da ciência e tecnologia}
\setcounter{page}{46}
\chapter{Caminhantes noturnos e vendedores de mistérios: senso e contra-senso às margens da ciência} 
\setcounter{page}{57}
\chapter{Anãs Brancas e homenzinhos verdes}
\setcounter{page}{81}
\chapter{Vênus e o Dr. Velikovsky}
\setcounter{page}{96}
\chapter{Normam Bloom, mensageiro de Deus}
\setcounter{page}{144}
\chapter{Ficção científica -- uma visão pessoal}
\setcounter{page}{153}




\end{document}
