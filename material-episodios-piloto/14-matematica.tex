% Cabeçalho para artigo

\documentclass[12pt,article,a5paper]{memoir}

% Escrevendo em português e utf8
\usepackage[utf8]{inputenc}
\usepackage[T1]{fontenc}
\usepackage[brazil]{babel}

\usepackage{amsmath}
\usepackage{amssymb}
\usepackage{amsthm}

% Novos operadores matemáticos
\DeclareMathOperator{\sen}{sen}

\begin{document}

\chapter{Dois jeitos de escrever fórmulas}

Uma formula é \(a+b+c+10\), a chamada
fórmula "na linha" (inline), porque fica
no meio do texto. Compare com \[a+b+c+10,\]
o chamado modo "destaque".
Uma formula é \(a + b  +  c      +10\).

\chapter{Algumas expressões básicas}

\begin{itemize}
\item soma: \(a + b + c\);
\item subtração: \(a + b - c\);
\item multiplicação: \(a + b c\);
\item produto: \(a + b \times c\);
\item divisão: \(a + b / c\);
\item divisão: \(a + b \div c\).
\end{itemize}

"Reticências" para fórmulas:

Os números naturais são~\(1,2,3,\ldots\)

A soma desses números é~\(1+2+3+\cdots\)

Fração com dois andares: \(\frac{a}{b}\)

Fração com dois andares: \(10 + \frac{a+b}{bc}\)

Fração com dois andares: \[10 + \frac{a+b}{bc}\]

Frações mais complicadas
\[
10 + \frac{a+b}{bc + \frac{2}{3+c}}
\]

% sqrt = square root = raiz quadrada
Raízes~\(\sqrt{2}\) é irracional.

Outras:~\(\sqrt[3+a]{20}\)

\[
\sqrt[2/(3+a)]{a+bc}
\]

\section{Sobrescrito}

\[
(a+bc)^2
\]

\[
(a+bc)^{23}
\]

\[
(a+bc)^{\frac{1}{2/(3+a)}}
\]

Com conjuntos, \(A^*\) 
é diferente de \(A*\).

\section{Subscrito}

Coeficientes: \(a_1, b_2, \ldots\)

Mais complicado
\[
a_{3b + 1} + 10
\]
é diferente de 
\[
a_3b + 1
\]

\chapter{Relações entre expressões}

Se \(0 < a < b\), então
\[
a^2 < b^2.
\]
Por outro lado, se~\(a < b < 0\), então
\[
a^2 > b^2.
\]

% menor ou igual: less than or equal <=
Se~\(a \leq b\),
% maior ou igual: greater than or equal >=
e~\(a \geq b\),
então
\(a=b\).

% diferente: not equal
Se~\(a<b\) ou~\(a>b\) então~\(a \neq b\).

\clearpage
\section{Operadores}

\begin{itemize}
\item seno \(\sin x\)
\item cosseno \(\cos x\)
\item logaritmo sem base \(\log x\)
\item logaritmo natural \(\ln x\)
\item logaritmo com base \( \log_{10} x\)
\end{itemize}

\section{novos operadores}

Use o comando DeclareMathOperator 
(veja preâmbulo): \(\sen x\)

\chapter{Implicações}

Se \(A\), então~\(B\).
\[
A \implies B
\]
Se e somente se
\[
A \iff B
\]

\chapter{Flechas}

Seja~\(f\) uma função dos naturais nos reais
\[
f\colon N \to R
\]
\[
f: N \to R
\]

\chapter{Somas e limites}

\[
1+2+\cdots + n = \frac{n(n-1)}{2}.
\]
ou
\[
\sum_{i=1}^n i = \frac{n(n-1)}{2}
\]

Matemática amadora \((fg)' = f'g + fg'\)
\[
\int_a^b fg' = fg - \int f'g
\]

Limites~\(\lim_{x\to 0} f(x) = C\).
\[
\lim_{x\to 0} f(x) = C.
\]
\end{document}



se, então

contido, contém
pertence

somas e integrais







