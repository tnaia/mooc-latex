\documentclass[12pt]{memoir}

\usepackage[utf8]{inputenc}
\usepackage[T1]{fontenc}
\usepackage[brazil]{babel}

\begin{document}



Hoje vamos discutir alinhamento de texto,
nas três modalidades mais comuns: texto justificado, centralizado, alinhado à esquerda e alinhado à direita.

\begin{center}
Algumas linhas centralizadas.
Algumas linhas centralizadas.
Algumas linhas centralizadas.
Algumas linhas centralizadas.

Algumas linhas centralizadas.
Algumas linhas centralizadas.
Algumas linhas centralizadas.

Se essa rua, se essa rua fosse minha \\ 
eu mandava, eu mandava ladrilhar \\
com pedrinhas, com pedrinhas e brilhantes \\
para o meu, só pro meu amor passar.
\end{center}


Se queremos texto alinhado à esquerda,
usamos flushleft, como você verá abaixo.


\begin{flushleft}
Título: Os Lusíadas

Autor: Camões

Data: Muito tempo atrás.
\end{flushleft}

Para alinhar o texto à direita, use o ambiente flushright. (blablabla blablabla blablabla blablabla blablabla blablabla )

\begin{flushright}
Título: Os Lusíadas

Autor: Camões

Data: Muito tempo atrás.
\end{flushright}


\end{document}