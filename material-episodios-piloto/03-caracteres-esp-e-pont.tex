\documentclass[12pt]{memoir}

\usepackage[utf8]{inputenc}
\usepackage[T1]{fontenc}
\usepackage[brazil]{babel}

\begin{document}

Pontuação


Traços

Os dias --- na verdade, os anos ---
que passei esperando chuva foram
longos.

--- Claro que não, disse José.

A loja abre entre 10--11 horas.

A ponte Rio--Niterói.

Guarda-chuva



Aspas.

`texto entre aspas simples'

``texto entre aspas duplas''

Hoje levantei ``cedo''.


Quebra de linha


Hoje levantei tão cedo que decidi voltar para a cama. \newline
Depois, quando vi, já eram 10 horas!

Hoje levantei tão cedo que decidi voltar para a cama. \\
Depois, quando vi, já eram 10 horas!




% Agora vamos falar sobre comentários
% usando o caractere %

% Um comentário muito longo, que passa
% (aparentemente) para a próxima linha


Comentários % como este aqui


A vida marinha\ldots
% Falar sobre baleias?
é muito agradável, e molhada.



A vida marinha é muito agradável, e molhada.

% A vida marinha é muito agradável, apesar molhada.



Caracteres especiais


Sabia que 30% das pessoas são felizes?

para a barra, \textbackslash \newline
para o porcento, \%          \newline
para o `e' comercial, \&     \newline
para o cifrão, \$            \newline
para o jogo da velha ,\#     \newline
para o underline, \_         \newline
para abrir chaves, \{        \newline
para fechar chaves, \}       \newline
para o til, \~               \newline
para o circunflexo \^        \newline


Lshort (uma não tão curta introdução
ao \LaTeX), 
comprehensive latex symbol list




















\newpage

\noindent 
\`{o} --- acento grave        \\
\'{o} --- acento agudo        \\
\^{o} --- acento circunflexo  \\
\"{o} --- trema               \\
\H{o} --- acento longo Húngaro (acento agudo duplo)                  \\
\~{o} --- til                 \\
\c{c} --- cedilha             \\
\k{a} --- ogonek              \\
\l{}  --- l barrado           \\
\={o} --- acento mácron	    \\
\b{o} --- barra sob  a letra  \\
\.{o} --- ponto sobre a letra \\
\d{u} --- ponto sob a letra   \\
\r{a} --- anel (pequeno círculo sobre a letra)             \\
\u{o} --- bráquia  \\
\v{s} --- caron (pequeno ``v'' sobre a letra)             \\
\t{oo} --- ``arco'' (``u'' invertido sobre duas letras) \\
\o --- o barrado   \\

Acentos sobre ``i'' e ``j'': 
remova o pingo, com \i\ e \j.
Por exemplo,
\^\i para  'î';
\"\i para 'ï'.

(comentar utf8).





\end{document}