\documentclass[10pt,landscape]{article}
\usepackage[utf8]{inputenc}
\usepackage[T1]{fontenc}
\usepackage[brazil]{babel}
\usepackage{lmodern,microtype}
\usepackage{multicol}
\usepackage{calc}
\usepackage{ifthen}
\usepackage[landscape]{geometry}
\usepackage{hyperref}

% To make this come out properly in landscape mode, do one of the following
% 1.
%  pdflatex latexsheet.tex
%
% 2.
%  latex latexsheet.tex
%  dvips -P pdf  -t landscape latexsheet.dvi
%  ps2pdf latexsheet.ps


% If you're reading this, be prepared for confusion.  Making this was
% a learning experience for me, and it shows.  Much of the placement
% was hacked in; if you make it better, let me know...


% 2008-04
% Changed page margin code to use the geometry package. Also added code for
% conditional page margins, depending on paper size. Thanks to Uwe Ziegenhagen
% for the suggestions.

% 2006-08
% Made changes based on suggestions from Gene Cooperman. <gene at ccs.neu.edu>


% To Do:
% \listoffigures \listoftables
% \setcounter{secnumdepth}{0}


% This sets page margins to .5 inch if using letter paper, and to 1cm
% if using A4 paper. (This probably isn't strictly necessary.)
% If using another size paper, use default 1cm margins.
\ifthenelse{\lengthtest { \paperwidth = 11in}}
	{ \geometry{top=.5in,left=.5in,right=.5in,bottom=.5in} }
	{\ifthenelse{ \lengthtest{ \paperwidth = 297mm}}
		{\geometry{top=1cm,left=1cm,right=1cm,bottom=1cm} }
		{\geometry{top=1cm,left=1cm,right=1cm,bottom=1cm} }
	}

% Turn off header and footer
\pagestyle{empty}
 

% Redefine section commands to use less space
\makeatletter
\renewcommand{\section}{\@startsection{section}{1}{0mm}%
                                {-1ex plus -.5ex minus -.2ex}%
                                {0.5ex plus .2ex}%x
                                {\normalfont\large\bfseries}}
\renewcommand{\subsection}{\@startsection{subsection}{2}{0mm}%
                                {-1explus -.5ex minus -.2ex}%
                                {0.5ex plus .2ex}%
                                {\normalfont\normalsize\bfseries}}
\renewcommand{\subsubsection}{\@startsection{subsubsection}{3}{0mm}%
                                {-1ex plus -.5ex minus -.2ex}%
                                {1ex plus .2ex}%
                                {\normalfont\small\bfseries}}
\makeatother

% Define BibTeX command
\def\BibTeX{{\rm B\kern-.05em{\sc i\kern-.025em b}\kern-.08em
    T\kern-.1667em\lower.7ex\hbox{E}\kern-.125emX}}

% Don't print section numbers
\setcounter{secnumdepth}{0}


\setlength{\parindent}{0pt}
\setlength{\parskip}{0pt plus 0.5ex}


% -----------------------------------------------------------------------

\begin{document}

\raggedright
\footnotesize
\begin{multicols}{3}


% multicol parameters
% These lengths are set only within the two main columns
%\setlength{\columnseprule}{0.25pt}
\setlength{\premulticols}{1pt}
\setlength{\postmulticols}{1pt}
\setlength{\multicolsep}{1pt}
\setlength{\columnsep}{2pt}

\begin{center}
     \Large{\textbf{Resumão \LaTeXe}} \\
\end{center}

\section{Classes de documento}
\begin{tabular}{@{}ll@{}}
\verb!book!    & Frente e verso por padrão. \\
\verb!report!  & Sem divisão em \verb!\part!. \\
\verb!article! & Sem \verb!\part! ou \verb!\chapter!. \\
\verb!letter!  & Carta. \\
\verb!slides!  & Fontes grandes e sem serifa.
\end{tabular}

No início do documento:
\verb!\documentclass{!\textit{class}\verb!}!.  Use
\verb!\begin{document}! marca o começo do conteúdo,
que termina em \verb!\end{document}!.


\subsection{Opções comuns de \texttt{documentclass}}
\newlength{\MyLen}
\settowidth{\MyLen}{\texttt{letterpaper}/\texttt{a4paper} \ }
\begin{tabular}{@{}p{\the\MyLen}%
                @{}p{\linewidth-\the\MyLen}@{}}
\texttt{10pt}/\texttt{11pt}/\texttt{12pt} & Tamanho da fonte. \\
\texttt{letterpaper}/\texttt{a4paper} & Tamanho do papel. \\
\texttt{twocolumn} & Use duas colunas. \\
\texttt{twoside}   & Margens para impressão frente e verso. \\
\texttt{landscape} & Página formato paisagem. Requer\newline
                     \texttt{dvips -t landscape}. \\
\texttt{draft}     & Espaçamento duplo.
\end{tabular}

Uso:
\verb!\documentclass[!\textit{opt,opt}\verb!]{!\textit{class}\verb!}!.


\subsection{Pacotes}
\settowidth{\MyLen}{\texttt{multicol} }
\begin{tabular}{@{}p{\the\MyLen}%
                @{}p{\linewidth-\the\MyLen}@{}}
%\begin{tabular}{@{}ll@{}}
\texttt{fullpage}  &  Margem de uma polegada. \\
\texttt{anysize}   &  Escolha margens: \verb!\marginsize{!\textit{l}%
                        \verb!}{!\textit{r}\verb!}{!\textit{t}%
                        \verb!}{!\textit{b}\verb!}!.            \\
\texttt{multicol}  &  Use $n$ colunas: 
                        \verb!\begin{multicols}{!$n$\verb!}!.   \\
\texttt{latexsym}  &  Use fonte \LaTeX\ symbol. \\
\texttt{graphicx}  &  Para figuras:
                        \verb!\includegraphics[width=!%
                        \textit{x}\verb!]{!%
                        \textit{arquivo}\verb!}!. \\
\texttt{url}       & Para links: \verb!\url{!%
                        \textit{http://\ldots}%
                        \verb!}!.
\end{tabular}

Use antes de  \verb!\begin{document}!. 
Uso: \verb!\usepackage{!\textit{package}\verb!}!


\subsection{Título}
\settowidth{\MyLen}{\texttt{.author.text.} }
\begin{tabular}{@{}p{\the\MyLen}%
                @{}p{\linewidth-\the\MyLen}@{}}
\verb!\author{!\textit{text}\verb!}! & Autor. \\
\verb!\title{!\textit{text}\verb!}!  & Título. \\
\verb!\date{!\textit{text}\verb!}!   & Data. \\
\end{tabular}

Estes comandos vêm antes de \verb!\begin{document}!.  
O comando
\verb!\maketitle! insere o título no documento.

\subsection{Outros}
\settowidth{\MyLen}{\texttt{.pagestyle.empty.} }
\begin{tabular}{@{}p{\the\MyLen}%
                @{}p{\linewidth-\the\MyLen}@{}}
\verb!\pagestyle{empty}!     &   Sem cabeçalho, rodapé
                                 ou número da página. \\
\verb!\tableofcontents!      &   Insere sumário (índice). \\

\end{tabular}



\section{Estrutura do documento}
\begin{multicols}{2}
\verb!\part{!\textit{title}\verb!}!  \\
\verb!\chapter{!\textit{title}\verb!}!  \\
\verb!\section{!\textit{title}\verb!}!  \\
\verb!\subsection{!\textit{title}\verb!}!  \\
\verb!\subsubsection{!\textit{title}\verb!}!  \\
\verb!\paragraph{!\textit{title}\verb!}!  \\
\verb!\subparagraph{!\textit{title}\verb!}!
\end{multicols}
{\raggedright
Use \verb!\setcounter{secnumdepth}{!$x$\verb!}! para omitir
seções com profundidade maior que~$x$:  \verb!chapter! tem profundidade 0.
Use um \texttt{*}, p.\,ex.\ \verb!\section*{!\textit{title}\verb!}!
para divisão não-numerada---ela não vai aparecer no índice.
}

\subsection{Ambientes}
\settowidth{\MyLen}{\texttt{.begin.quotation.}}
\begin{tabular}{@{}p{\the\MyLen}%
                @{}p{\linewidth-\the\MyLen}@{}}
\verb!\begin{comment}!    &  Commentário (não é impresso). Requer o pacote \texttt{verbatim}. \\
\verb!\begin{quote}!      &  Citação. \\
\verb!\begin{quotation}!  &  Citação como \texttt{quote}, com parágrafos. \\
\verb!\begin{verse}!      &  Bloco de citação para versos.
\end{tabular}

\subsection{Listas}
\settowidth{\MyLen}{\texttt{.begin.description.}}
\begin{tabular}{@{}p{\the\MyLen}%
                @{}p{\linewidth-\the\MyLen}@{}}
\verb!\begin{enumerate}!        &  Numerada. \\
\verb!\begin{itemize}!          &  Com pontos. \\
\verb!\begin{description}!      &  Descrição. \\
\verb!\item! \textit{text}      &  Item da lista. \\
\verb!\item[!\textit{x}\verb!]! \textit{text}
                                &  Use \textit{x} em vez do ponto ou número normal.  
                                   Obrigatório em descrições. \\
\end{tabular}




\subsection{Refências}
\settowidth{\MyLen}{\texttt{.pageref.marker..}}
\begin{tabular}{@{}p{\the\MyLen}%
                @{}p{\linewidth-\the\MyLen}@{}}
\verb!\label{!\textit{marker}\verb!}!   &  Define rótulo para referência cruzada,\newline
                          p.\,ex.\ \verb!\label{sec:item}!. \\
\verb!\ref{!\textit{marker}\verb!}!   &  Número associado ao rótulo. \\
\verb!\pageref{!\textit{marker}\verb!}! &  Página em que o rótulo ocorreu. \\
\verb!\footnote{!\textit{text}\verb!}!  &  Nota de rodapé. \\
\end{tabular}


\subsection{Ambientes flutuantes}
\settowidth{\MyLen}{\texttt{.begin.equation..place.}}
\begin{tabular}{@{}p{\the\MyLen}%
                @{}p{\linewidth-\the\MyLen}@{}}
\verb!\begin{table}[!\textit{posição}\verb!]!     &  Tabela numerada. \\
\verb!\begin{figure}[!\textit{posição}\verb!]!    &  Figura numerada. \\
\verb!\begin{equation}[!\textit{posição}\verb!]!  &  Equação numerada. \\
\verb!\caption{!\textit{legenda}\verb!}!           &  Legenda para o ambiente. \\
\end{tabular}

A \textit{posição} é uma lista com preferências para o posicionamento. 
\texttt{t}=topo, \texttt{h}=aqui (here), \texttt{b}=base, 
\texttt{p}=página separada, \texttt{!}=coloque mesmo se achar feio.
Legenda e rótulo (\verb!\label!) ficam dentro do ambiente.

%---------------------------------------------------------------------------

\section{Propriedades do texto}

\subsection{Estilo da fonte}
\newcommand{\FontCmd}[3]{\PBS\verb!\#1{!\textit{text}\verb!}!  \> %
                         \verb!{\#2 !\textit{text}\verb!}! \> %
                         \#1{#3}}
\begin{tabular}{@{}l@{}l@{}l@{}}
\textit{Commando} & \textit{Declaração} & \textit{Efeito} \\
\verb!\textrm{!\textit{text}\verb!}!                    & %
        \verb!{\rmfamily !\textit{text}\verb!}!               & %
        \textrm{Fam.\ romana} \\
\verb!\textsf{!\textit{text}\verb!}!                    & %
        \verb!{\sffamily !\textit{text}\verb!}!               & %
        \textsf{Fam.\ Sem Serifa} \\
\verb!\texttt{!\textit{text}\verb!}!                    & %
        \verb!{\ttfamily !\textit{text}\verb!}!               & %
        \texttt{Fam. Datilográfica} \\
\verb!\textmd{!\textit{text}\verb!}!                    & %
        \verb!{\mdseries !\textit{text}\verb!}!               & %
        \textmd{Peso Médio} \\
\verb!\textbf{!\textit{text}\verb!}!                    & %
        \verb!{\bfseries !\textit{text}\verb!}!               & %
        \textbf{Peso Negrito} \\
\verb!\textup{!\textit{text}\verb!}!                    & %
        \verb!{\upshape !\textit{text}\verb!}!               & %
        \textup{Forma Vertical} \\
\verb!\textit{!\textit{text}\verb!}!                    & %
        \verb!{\itshape !\textit{text}\verb!}!               & %
        \textit{Forma Itálica} \\
\verb!\textsl{!\textit{text}\verb!}!                    & %
        \verb!{\slshape !\textit{text}\verb!}!               & %
        \textsl{Forma Inclinada} \\
\verb!\textsc{!\textit{text}\verb!}!                    & %
        \verb!{\scshape !\textit{text}\verb!}!               & %
        \textsc{Forma Versalete} \\
\verb!\emph{!\textit{text}\verb!}!                      & %
        \verb!{\em !\textit{text}\verb!}!               & %
        \emph{Enfatizado} \\
\verb!\textnormal{!\textit{text}\verb!}!                & %
        \verb!{\normalfont !\textit{text}\verb!}!       & %
        \textnormal{Fonte padrão} \\
\verb!\underline{!\textit{text}\verb!}!                 & %
                                                        & %
        \underline{Sublinhado}
\end{tabular}

Para melhor espaçamento use comando (p.\,ex.\ t\textit{tt}t) 
em vez de declaração (p.\,ex.\ t{\itshape tt}t).

\subsection{Font size}
\setlength{\columnsep}{14pt} % Need to move columns apart a little
\begin{multicols}{2}
\begin{tabbing}
\verb!\footnotesize!          \= \kill
\verb!\tiny!                  \>  \tiny{tiny} \\
\verb!\scriptsize!            \>  \scriptsize{scriptsize} \\
\verb!\footnotesize!          \>  \footnotesize{footnotesize} \\
\verb!\small!                 \>  \small{small} \\
\verb!\normalsize!            \>  \normalsize{normalsize} \\
\verb!\large!                 \>  \large{large} \\
\verb!\Large!                 \=  \Large{Large} \\  % Tab hack for new column
\verb!\LARGE!                 \>  \LARGE{LARGE} \\
\verb!\huge!                  \>  \huge{huge} \\
\verb!\Huge!                  \>  \Huge{Huge}
\end{tabbing}
\end{multicols}
\setlength{\columnsep}{1pt} % Set column separation back

\smallskip
Estas são declarações. Use na forma
\verb!{\small! \ldots\verb!}!, ou sem as chaves para afetar o documento 
todo a partir desse ponto.


\subsection{Texto verbatim}

\settowidth{\MyLen}{\texttt{.begin.verbatim..} }
\begin{tabular}{@{}p{\the\MyLen}%
                @{}p{\linewidth-\the\MyLen}@{}}
\verb@\begin{verbatim}@ & Ambiente verbatim. \\
\verb@\begin{verbatim*}@ & Espaços são mostrados assim \verb*@ @. \\
\verb@\verb!text!@ & O texto entre os caracteres delimidatos (neste caso %
                      `\texttt{!}') em verbatim.
\end{tabular}


\subsection{Alinhamento}
\begin{tabular}{@{}ll@{}}
\textit{Ambiente}  &  \textit{Declaração}  \\
\verb!\begin{center}!      & \verb!\centering!  \\
\verb!\begin{flushleft}!  & \verb!\raggedright! \\
\verb!\begin{flushright}! & \verb!\raggedleft!  \\
\end{tabular}

\subsection{Outros}
\verb!\linespread{!$x$\verb!}! muda o espaçamento da linha de 
um~múltiplo~$x$.





\section{Símbolos no modo texto}

\subsection{Símbolos}
\begin{tabular}{@{}l@{\hspace{1em}}l@{\hspace{2em}}l@{\hspace{1em}}l@{\hspace{2em}}l@{\hspace{1em}}l@{\hspace{2em}}l@{\hspace{1em}}l@{}}
\&              &  \verb!\&! &
\_              &  \verb!\_! &
\ldots          &  \verb!\ldots! &
\textbullet     &  \verb!\textbullet! \\
\$              &  \verb!\$! &
\^{}            &  \verb!\^{}! &
\textbar        &  \verb!\textbar! &
\textbackslash  &  \verb!\textbackslash! \\
\%              &  \verb!\%! &
\~{}            &  \verb!\~{}! &
\#              &  \verb!\#! &
\S              &  \verb!\S! \\
\end{tabular}

\subsection{Acentos}
\begin{tabular}{@{}l@{\ }l|l@{\ }l|l@{\ }l|l@{\ }l|l@{\ }l@{}}
\`o   & \verb!\`o! &
\'o   & \verb!\'o! &
\^o   & \verb!\^o! &
\~o   & \verb!\~o! &
\=o   & \verb!\=o! \\
\.o   & \verb!\.o! &
\"o   & \verb!\"o! &
\c o  & \verb!\c o! &
\v o  & \verb!\v o! &
\H o  & \verb!\H o! \\
\c c  & \verb!\c c! &
\d o  & \verb!\d o! &
\b o  & \verb!\b o! &
\t oo & \verb!\t oo! &
\oe   & \verb!\oe! \\
\OE   & \verb!\OE! &
\ae   & \verb!\ae! &
\AE   & \verb!\AE! &
\aa   & \verb!\aa! &
\AA   & \verb!\AA! \\
\o    & \verb!\o! &
\O    & \verb!\O! &
\l    & \verb!\l! &
\L    & \verb!\L! &
\i    & \verb!\i! \\
\j    & \verb!\j! &
!`    & \verb!~`! &
?`    & \verb!?`! &
\end{tabular}


\subsection{Delimitadores}
\begin{tabular}{@{}l@{\ }ll@{\ }ll@{\ }ll@{\ }ll@{\ }ll@{\ }l@{}}
`       & \verb!`!  &
``      & \verb!``! &
\{      & \verb!\{! &
\lbrack & \verb![! &
(       & \verb!(! &
\textless  &  \verb!\textless! \\
'       & \verb!'!  &
''      & \verb!''! &
\}      & \verb!\}! &
\rbrack & \verb!]! &
)       & \verb!)! &
\textgreater  &  \verb!\textgreater! \\
\end{tabular}

\subsection{Traços}
\begin{tabular}{@{}llll@{}}
\textit{Nome} & \textit{Source} & \textit{Example} & \textit{Usage} \\
hífem  & \verb!-!   & raio-X          & Em palavras. \\
traço ene & \verb!--!  & 1--5, Rio--Niterói   & Entre números. \\
traço eme & \verb!---! & Sim---ou não?    & Pontuação.
\end{tabular}


\subsection{Linhas e quebras de página}
\settowidth{\MyLen}{\texttt{.pagebreak} }
\begin{tabular}{@{}p{\the\MyLen}%
                @{}p{\linewidth-\the\MyLen}@{}}
\verb!\\! e \verb!\newline! &  Nova linha, mesmo parágrafo.  \\
\verb!\\*!                  &  Nova linha mas proíbe mudar de página. \\
\verb!\kill!                &  Não imprima a linha atual. \\
\verb!\pagebreak!           &  Nova página. \\
\verb!\noindent!            &  Linha sem recuo (i.e., sem indentação).
\end{tabular}


\subsection{Miscelânea}
\settowidth{\MyLen}{\texttt{.rule.w..h.} }
\begin{tabular}{@{}p{\the\MyLen}%
                @{}p{\linewidth-\the\MyLen}@{}}
\verb!\today!  &  \today (data de hoje). \\
\verb!$\sim$!  &  Imprime $\sim$ em vez de \verb!\~{}!, que produz \~{}. \\
\verb!~!       &  Espaço que sem quebra de linha (\verb!M. de~Assis!). \\
\verb!\@.!     &  Indica que o ponto termina uma frase. \\
\verb!\hspace{!$x$\verb!}! & Espaço horizontal medindo~$x$. \\
\verb!\vspace{!$x$\verb!}! & Espaço vertical medindo~$x$. \\
\verb!\rule{!$\ell$\verb!}{!$a$\verb!}! & Linha de largura $\ell$ e altura $a$. \\
\end{tabular}



\section{Ambientes tabulares}

\subsection{Ambiente \texttt{tabbing}}
\begin{tabular}{@{}l@{\hspace{1.5ex}}l@{\hspace{10ex}}l@{\hspace{1.5ex}}l@{}}
\verb!\=!  &   Nova tabulação. &
\verb!\>!  &   Próxima tabulação.
\end{tabular}

Tabulações podem ser criadas em uma linha invisível, usando \verb!\kill!
no fim dalinha.  Use \verb!\\! ou \verb|\newline| para mudar de linha.


\subsection{Ambiente \texttt{tabular}}
\verb!\begin{array}[!\textit{posição}\verb!]{!\textit{colunas}\verb!}!   \\
\verb!\begin{tabular}[!\textit{posição}\verb!]{!\textit{colunas}\verb!}! \\
\verb!\begin{tabular*}{!\textit{largura}\verb!}[!\textit{posição}\verb!]{!\textit{colunas}\verb!}!


\subsubsection{Colunas de \texttt{tabular}}
\settowidth{\MyLen}{\texttt{p}\{\textit{largura}\} \ }
\begin{tabular}{@{}p{\the\MyLen}@{}p{\linewidth-\the\MyLen}@{}}
\texttt{l}    &   Alinhada à esquerda.  \\
\texttt{c}    &   Centralizada.  \\
\texttt{r}    &   Alinhada à direita. \\
\verb!p{!\textit{largura}\verb!}!  &  É o mesmo que %
                              \verb!\parbox[t]{!\textit{largura}\verb!}!. \\ 
\verb!@{!\textit{texto}\verb!}!   &  Insere \textit{texto} em vez do espaço
                                    entre colunas. \\
\verb!|!      &   Barra vertical entre colunas. 
\end{tabular}


\subsubsection{Elementos em \texttt{tabular}}
\settowidth{\MyLen}{\texttt{.cline.x-y..}}
\begin{tabular}{@{}p{\the\MyLen}@{}p{\linewidth-\the\MyLen}@{}}
\verb!\hline!           &  Linha horizontal.  \\
\verb!\cline{!$x$\verb!-!$y$\verb!}!  &
                        Idem, desde a coluna $x$ até a $y$. \\
\verb!\multicolumn{!\textit{n}\verb!}{!\textit{cols}\verb!}{!\textit{text}\verb!}! \\
        &  Célula combinando \textit{n} colunas, e alinhamento \textit{cols}.
\end{tabular}

\section{Modo matemático}
Para matemática na linha use \verb!\(...\)! ou \verb!$...$!.
Para destaque use \verb!\[...\]! ou \verb!\begin{equation}!.

\begin{tabular}{@{}l@{\hspace{1em}}l@{\hspace{2em}}l@{\hspace{1em}}l@{}}
Expoente$^{x}$       &
\verb!^{x}!             &  
Índice$_{x}$         &
\verb!_{x}!             \\  
$\frac{x}{y}$           &
\verb!\frac{x}{y}!      &  
$\sum_{k=1}^n$          &
\verb!\sum_{k=1}^n!     \\  
$\sqrt[n]{x}$           &
\verb!\sqrt[n]{x}!      &  
$\prod_{k=1}^n$         &
\verb!\prod_{k=1}^n!    \\ 
\end{tabular}

\subsection{Símbolos no modo matemático}

% The ordering of these symbols is slightly odd.  This is because I had to put all the
% long pieces of text in the same column (the right) for it all to fit properly.
% Otherwise, it wouldn't be possible to fit four columns of symbols here.

\begin{tabular}{@{}l@{\hspace{1ex}}l@{\hspace{1em}}l@{\hspace{1ex}}l@{\hspace{1em}}l@{\hspace{1ex}} l@{\hspace{1em}}l@{\hspace{1ex}}l@{}}
$\leq$          &  \verb!\leq!  &
$\geq$          &  \verb!\geq!  &
$\neq$          &  \verb!\neq!  &
$\approx$       &  \verb!\approx!  \\
$\times$        &  \verb!\times!  &
$\div$          &  \verb!\div!  &
$\pm$           & \verb!\pm!  &
$\cdot$         &  \verb!\cdot!  \\
$^{\circ}$      & \verb!^{\circ}! &
$\circ$         &  \verb!\circ!  &
$\prime$        & \verb!\prime!  &
$\cdots$        &  \verb!\cdots!  \\
$\infty$        & \verb!\infty!  &
$\neg$          & \verb!\neg!  &
$\wedge$        & \verb!\wedge!  &
$\vee$          & \verb!\vee!  \\
$\supset$       & \verb!\supset!  &
$\forall$       & \verb!\forall!  &
$\in$           & \verb!\in!  &
$\rightarrow$   &  \verb!\rightarrow! \\
$\subset$       & \verb!\subset!  &
$\exists$       & \verb!\exists!  &
$\notin$        & \verb!\notin!  &
$\Rightarrow$   &  \verb!\Rightarrow! \\
$\cup$          & \verb!\cup!  &
$\cap$          & \verb!\cap!  &
$\mid$          & \verb!\mid!  &
$\Leftrightarrow$   &  \verb!\Leftrightarrow! \\
$\dot a$        & \verb!\dot a!  &
$\hat a$        & \verb!\hat a!  &
$\bar a$        & \verb!\bar a!  &
$\tilde a$      & \verb!\tilde a!  \\

$\alpha$        &  \verb!\alpha!  &
$\beta$         &  \verb!\beta!  &
$\gamma$        &  \verb!\gamma!  &
$\delta$        &  \verb!\delta!  \\
$\epsilon$      &  \verb!\epsilon!  &
$\zeta$         &  \verb!\zeta!  &
$\eta$          &  \verb!\eta!  &
$\varepsilon$   &  \verb!\varepsilon!  \\
$\theta$        &  \verb!\theta!  &
$\iota$         &  \verb!\iota!  &
$\kappa$        &  \verb!\kappa!  &
$\vartheta$     &  \verb!\vartheta!  \\
$\lambda$       &  \verb!\lambda!  &
$\mu$           &  \verb!\mu!  &
$\nu$           &  \verb!\nu!  &
$\xi$           &  \verb!\xi!  \\
$\pi$           &  \verb!\pi!  &
$\rho$          &  \verb!\rho!  &
$\sigma$        &  \verb!\sigma!  &
$\tau$          &  \verb!\tau!  \\
$\upsilon$      &  \verb!\upsilon!  &
$\phi$          &  \verb!\phi!  &
$\chi$          &  \verb!\chi!  &
$\psi$          &  \verb!\psi!  \\
$\omega$        &  \verb!\omega!  &
$\Gamma$        &  \verb!\Gamma!  &
$\Delta$        &  \verb!\Delta!  &
$\Theta$        &  \verb!\Theta!  \\
$\Lambda$       &  \verb!\Lambda!  &
$\Xi$           &  \verb!\Xi!  &
$\Pi$           &  \verb!\Pi!  &
$\Sigma$        &  \verb!\Sigma!  \\
$\Upsilon$      &  \verb!\Upsilon!  &
$\Phi$          &  \verb!\Phi!  &
$\Psi$          &  \verb!\Psi!  &
$\Omega$        &  \verb!\Omega!  
\end{tabular}
\footnotesize

%\subsection{Special symbols}
%\begin{tabular}{@{}ll@{}}
%$^{\circ}$  &  \verb!^{\circ}! Ex: $22^{\circ}\mathrm{C}$: \verb!$22^{\circ}\mathrm{C}$!.
%\end{tabular}

\section{Bibliografia e citações}
Se usar  \BibTeX, precisa compilar o documento mais de uma vez: 
\texttt{(pdf)latex}, \texttt{bibtex},
\texttt{(pdf)latex} e \texttt{(pdf)latex} de novo.

\subsection{Tipos de citações}
\settowidth{\MyLen}{\texttt{.shortciteN.key..}}
\begin{tabular}{@{}p{\the\MyLen}@{}p{\linewidth-\the\MyLen}@{}}
\verb!\cite{!\textit{key}\verb!}!       &
        Lista de autores e ano. (Watson and Crick 1953) \\
\verb!\citeA{!\textit{key}\verb!}!      &
        Somente os autores. (Watson and Crick) \\
\verb!\citeN{!\textit{key}\verb!}!      &
        Autores e ano. Watson and Crick (1953) \\
\verb!\shortcite{!\textit{key}\verb!}!  &
        Autor (abreviado) e ano. ? \\
\verb!\shortciteA{!\textit{key}\verb!}! &
        Autor (abreviado). ? \\
\verb!\shortciteN{!\textit{key}\verb!}! &
        Autor (abreviado) e ano. ? \\
\verb!\citeyear{!\textit{key}\verb!}!   &
        Ano apenas. (1953) \\
\end{tabular}

Todos acima possuem uma variante \texttt{NP} sem parênteses,
p\,ex.\ \verb!\citeNP!.


\subsection{Tipos de entradas bibliográficas (\BibTeX)}
\settowidth{\MyLen}{\texttt{.mastersthesis.}}
\begin{tabular}{@{}p{\the\MyLen}@{}p{\linewidth-\the\MyLen}@{}}
\verb!@article!         &  Artigo em periódico. \\
\verb!@book!            &  Livro com editora. \\
\verb!@booklet!         &  Livro sem editora. \\
\verb!@conference!      &  Artigo em anais de conferência. \\
\verb!@inbook!          &  Parte de um livro (intervalo de páginas). \\
\verb!@incollection!    &  Parte (com título) de um livro (p.\,ex., fascículo. \\
%\verb!@manual!          &  Technical documentation. \\
%\verb!@mastersthesis!   &  Master's thesis. \\
\verb!@misc!            &  Miscelânea (i.\,e. nenhuma das outras). \\
\verb!@phdthesis!       &  Tese de doutorado. \\
\verb!@proceedings!     &  Anais de uma conferência. \\
\verb!@techreport!      &  Relatório técnico, geralmente numerado. \\
\verb!@unpublished!     &  Documento que não foi publicado. \\
\end{tabular}

\subsection{Campos das entradas (\BibTeX)}
\settowidth{\MyLen}{\texttt{organization.}}
\begin{tabular}{@{}p{\the\MyLen}@{}p{\linewidth-\the\MyLen}@{}}
\verb!address!       &  Endereço da editora. (Não é necessário se é uma editora grande).  \\
\verb!author!        &  Nomes dos autores, no formato\ldots \\
\verb!booktitle!     &  Título do livro se a entrada é parte dele. \\
\verb!chapter!       &  Número do capítulo ou da seção. \\
\verb!edition!       &  Edição. \\
\verb!editor!        &  Nome dos editores. \\
\verb!institution!   &  Instituição responsável pelo relatório. \\
\verb!journal!       &  Nome do periódico. \\
\verb!key!           &  Usado para referências cruzadas quando não há autor. \\
\verb!month!         &  Mês da publicação (abreviação com 3 letras). \\
\verb!note!          &  Informação adicional. \\
\verb!number!        &  Número do periódico. \\
\verb!organization!  &  Organização responsável pela conferência. \\
\verb!pages!         &  Intervalo de páginas (p.\,ex.\ \verb!2,6,9--12!). \\
\verb!publisher!     &  Nome da editora. \\
\verb!school!        &  Nome da escola (para teses). \\
\verb!series!        &  Nome da série de livros. \\
\verb!title!         &  Título. \\
\verb!type!          &  Tipo de relatório, p.\,ex.\ ``Notas de pesquisa. \\
\verb!volume!        &  Volume do periódico ou livro. \\
\verb!year!          &  Ano da publicação. \\
\end{tabular}
Nem todos os campos são obrigatórios.

\subsection{Estilos comuns de bilbiografia (\BibTeX)}
\begin{tabular}{@{}l@{\hspace{1em}}l@{\hspace{3em}}l@{\hspace{1em}}l@{}}
\verb!abbrv!    &  Padrão &
\verb!abstract! &  \texttt{alpha} com pacote \texttt{abstract} \\
\verb!alpha!    &  Padrão &
\verb!apa!      &  APA \\
\verb!plain!    &  Padrão &
\verb!unsrt!    &  Sem ordem alfabética \\
\end{tabular}

O documento deve ter as duas linhas abaixo antes de
\verb!\end{document}!, onde \verb!bibfile.bib! é o nome do arquivo
\BibTeX\ com as entradas bibliográficas.
\begin{verbatim}
\bibliographystyle{plain}
\bibliography{bibfile}
\end{verbatim}

\subsection{Exemplo de arquivo \BibTeX}
As entradas bibliográficas ficam num arquivo \BibTeX\ chamado
\textit{file}\texttt{.bib}, que é processado com \verb!bibtex file!. 
\begin{verbatim}
@String{N = {Na\-ture}}
@Article{WC:1953,
  author  = {James Watson and Francis Crick},
  title   = {A structure for Deoxyribose Nucleic Acid},
  journal = N,
  volume  = {171},
  pages   = {737},
  year    = 1953
}
\end{verbatim}


\section{Exemplo de documento \LaTeX}
\begin{verbatim}
\documentclass[11pt]{article}
\usepackage[utf8]{inputenc}
\usepackage[T1]{fontenc}
\usepackage[brazil]{babel} % escrever em português
\title{Documento básico}
\author{Nome do autor}
\begin{document}
\maketitle

\section{Título da primeira seção}
\subsection*{Subseção sem número}
Texto \textbf{em negrito} e uma expressão $2+3^2 > 0$
\subsection{Subseção numerada}
Texto \emph{enfatizado} e uma citação~\cite{WC:1953}
(descoberta da estrutura do ADN).

Uma tabela:
\begin{table}[!th]
\begin{tabular}{|l|c|r|}
\hline
first  &  row  &  data \\
second &  row  &  data \\
\hline
\end{tabular}
\caption{Esta é a legenda.}
\label{ex:table}
\end{table}

Veja a Tabela~\ref{ex:table}.
\end{document}
\end{verbatim}


\rule{0.3\linewidth}{0.25pt}
\scriptsize

Copyright \copyright\ 2014 Winston Chang
\href{http://www.stdout.org/~winston/latex/}{http://www.stdout.org/$\sim$winston/latex/}\newline
Adaptado e traduzido em 2016 por T\'assio Naia.

\end{multicols}
\end{document}
