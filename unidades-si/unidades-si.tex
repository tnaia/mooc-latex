% Usando o pacote siunitx
% Exemplo de Joseph Wright
% disponível em http://tex.stackexchange.com/questions/2248/what-package-should-i-use-to-typeset-units
\documentclass[article]{memoir}
\usepackage{siunitx}
\sisetup{load-configurations = abbreviations}

% Escrever em português e com utf-8
\usepackage[utf8]{inputenc}
\usepackage[T1]{fontenc}
\usepackage[brazil]{babel}

% Citar o link de onde vem o material
\usepackage[hyphens]{url}

\begin{document}

\noindent{\LARGE Sistema Internacional de medidas}\hfill 23 de março de 2016

\medskip

\noindent Joseph Wright

\medskip

\noindent 25/07/2014: formatação do exemplo (Tássio Naia).


\bigskip

\begin{abstract}
  Exemplos de expressões usando unidades do sistema intenacional de medidas (SI).
\end{abstract}


\chapter*{Agradecimentos \& crédito}

Agradecemos a Joseph Wright, por criar o pacote \textsf{siunitx} e pelo
exemplo abaixo, que foi obtido de
{\footnotesize\url{http://tex.stackexchange.com/questions/2248/what-package-should-i-use-to-typeset-units}. A tradução e os erros são de responsabilidade
de~Tássio Naia.

\begin{itemize}
  \item \SI{100}{\degreeCelsius}
  \item \SI{3e5}{\km\per\s} 
     or \SI[per-mode = symbol]{3e5}{\km\per\s}
  \item \si{\newton\metre\squared\per\kilogram\squared} or 
    \si[per-mode = symbol]{\newton\metre\squared\per\kilogram\squared}
  \item \(  \SI{10}{\kilo\hertz} = \SI{1}{\per\second\tothe{4}} \)
  \item \SI[parse-numbers = false]{\sin(x)}{\metre}
\end{itemize}

\end{document}
