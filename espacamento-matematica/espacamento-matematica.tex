% fonte: http://tex.stackexchange.com/questions/25810/when-one-should-use-spacing-line-quad-or
\documentclass[11pt,article]{memoir}

\usepackage[utf8]{inputenc}
\usepackage[T1]{fontenc}
\usepackage[brazil]{babel}
\usepackage{amsmath}

\usepackage{url} % para links
\begin{document}

\noindent{\LARGE Espaçamento em matemática}\hfill 03 de abril de 2016

\medskip

\noindent 03/04/2016: primeira versão (Tássio Naia).

\bigskip

\begin{abstract}
  Algumas recomendações para o espaçamento em expressões matemáticas.
  Muitos desses espaçamentos são automaticamente colocados uma vez que
  se usa o pacote ``amsmath''. Veja o texto-fonte deste arquivo.
\end{abstract}


\chapter*{Agradecimentos \& crédito}

Agradecemos a Gonzalo Medina; este texto é uma tradução de
uma de suas respostas no site \url{tex.stachexchange.com}.
A~tradução e os~erros são nossa~responsabilidade.

\chapter{Nenhum espaço.}
Entre dois símbolos e entre um número e o símbolo que ele está multiplicando:
\[
  ab\qquad xy\qquad 2a\qquad 2xz\qquad 4aC
\]

Antes e depois de índices, expoentes, parênteses, colchetes, chaves e barras
verticais:
\[
  2x^2y_3z (x)y\qquad a\{b\}\qquad y[a]\qquad a\lvert x\rvert\qquad b\lVert y\rVert
\]

Em expressões com índice ou expoente:
\[
  \lim_{0\to a}\qquad a^{n-1}
\]

\chapter{Espaço fino}
Antes e depois de símbolos usados como verbos:
\[
  a \subseteq 2
\]

Antes e depois de símbolos usados como conjunção:
\[
  a +2
\]

Depois, mas não antes $+$, $-$, $\pm$, $\mp$ usados como adjetivo:
\[
  a= -2
\]

Antes de vírgulas ao descrever conjuntos, sequências de frações e
coordenadas:
\[ (a,b,c) \]

Antes e depois de símbolos de integração, somatório, produtório e
união:
\[
  a\int x\,\mathrm{d}y
\]
O espaço fino entre $x$ e~$\mathrm{d}y$ na expressão acima \emph{não} é inserido
automaticamente pelo~\TeX.

Antes e depois de operadores com nome:
\[
a \sin x \qquad \log 2.
\]
Exceções: se alguma dessas funções for precedida ou seguida de parênteses,
colchetes, chaves ou barras o espaço deve ser eliminado:
\[
a \sin \lvert x \rvert.
\]

Antes e depois de barras que aparecem sozinhas (sem par);
a mesma regra aplica para dois-pontos usados como um símbolo
matemático (e não como sinal de pontuação):
\[
  a \mid b
\]

Antes de índices à esquerda:
\[
a\,_2T_3
\]
O espaço fino entre $a$ e o índice que vêm a seguir (e que pertence ao'' $T$)
\emph{não} é inserido automaticamente pelo~\TeX.

Antes e depois de $\mathrm{d}s$, $\mathrm{d}x$, e combinações similares de $d$
e outros símbolos:
\[
  \int f(x)\,\mathrm{d}x\qquad \iiint f(x)\,\mathrm{d}r\,\mathrm{d}\theta\,\mathrm{d}\phi
\]

\chapter{Espaço largo}
Amtes de árênteses em congruências no texto: $z= a\pmod x$.

Antes de condições matemáticas no corpo do texto: $t_n\ (n=1,2,\ldots, p)$

\chapter{Quad ``eme''}
Entre uma sentença simbólica e uma expressão verbal
no modo destaque:
\[
  E_n(t) \to e^{-t}\quad\text{as }t\to\infty.
\]

Em torno de conjunções:
\[
x(a+b)\quad\text{or}\quad y(a-b).
\]

\chapter{Quad duplo-eme}
Entre duas igualdades ou desigualdades na mesma linha, em destaque:
\[
x^2 + y^2 = a^2,\qquad x-y=b.
\]

Entre uma sentença simbólica e uma condição:
\[
x^n + y^n = a^n\qquad (n = 1,2,\ldots p).
\]

\end{document}
