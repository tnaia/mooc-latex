% Darnerson Pereira de Sousa
% Modificado 2016-03-23
%
% Mudanças
% 15-03-22 - remover tabulações, gramática (tássio)
\documentclass[12pt]{memoir}
\usepackage[utf8]{inputenc}
\usepackage[T1]{fontenc}
\usepackage[brazil]{babel}

\newenvironment{Entrevista}{%
  \par\bigskip\noindent\textbf{Jornal}\par}{%
  \par\smallskip\noindent\textbf{Alessandro La Porta}\par}
\begin{document}

\title{Entrevista sobre Dengue!}
\maketitle

Com a chegada das chuvas e, com isso, o almento de risco de transmissão  da dengue em toda região,  o jornal mais bom jardim conversou com o cordenador de vigilância sanitária e ambiental de bom jardim, Alessandro Laporta, sobre como eviar a doença, sua forma de controle e orientações a populaçao no combate ao mosquito.\\\par
Veremos algumas perguntas e respostas e resposta da entrevista:\\\par


\begin{Entrevista}
Por que tivemos um verão diferente dos outros quanto a dengue?
\end{Entrevista}

Estávamos passando por um período de muito calor, mas com baixa quantidade de chuvas para encher os possíveis depósitos, por isso a dengue não se manifestou.\@ Esse fenômeno ocorreu em todo estado e em algumas regiões do país.

\begin{Entrevista}
Mas agora temos recebido muitas chuvas, o que isso significa?
\end{Entrevista}

Significa que a dengue pode surgir em nossa região e diante desse cenário precisamos estar preparados para evitar que o mosquito se reproduza e só consiguiremos com a ajuda de todos.

\begin{Entrevista}
De que forma o poder público tem atuado no combate a dengue?
\end{Entrevista}

Realizamos visitas domiciliares e tratamos os depósitos que não podem ser eliminados com larvicidas que durem entre dois e três meses.\ Mas só este trabalho não é suficiente, visto que o Agente de controle de edemias só tem capacidade de passar em cada domicílio uma vez a cada 30, 40 dias. Por isso é fundamental a ajuda de todos para maximizar o trabalho dos Agentes.

\begin{Entrevista}
De que forma a população pode contribuir?
\end{Entrevista}

Devemos manter nossas casas e quintais sem depósitos de água parada, porque é lá que os mosquitos põem ovos que darão origem a novos vetores da doença. nós da secretaria da saúde, pedimos a parceria do bom--jardineiro no sentido de vistoriar suas residências a procura de vasos de plantas, pneus, verificar se as caixas dágua estão devidamente tampadas, se as calhas não estão acumulando água por estarem entupidas e qualquer outra forma de depósito de água sem utilidade, dentro ou fora de casa.

\begin{Entrevista}
E quanto às piscinas, é preciso tirar a água?
\end{Entrevista}

Essa é uma pergunta muito boa, se a piscina é bem tratada, a água sempre renova, e se colocar cloro toda semana, não tem problema.\ porém, tem gente que não limpa. É esse o problema da dengue: Basta haver água parada para ter a doença. O combate a dengue depende de todos nós, não apenas do governo.



\end{document}
