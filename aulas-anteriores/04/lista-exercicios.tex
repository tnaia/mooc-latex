% Projeto referente ao episódio 4: mais comandos, e um pouco de matemática. 
% Autoria: Michelle Sena
% Modificado 2016-03-23
%
% Mudanças
% 15-03-22 - 
\documentclass[12pt]{memoir}

\usepackage[utf8]{inputenc}
\usepackage[T1]{fontenc}
\usepackage[brazil]{babel}

\title{\textbf{\HUGE {Lista de Exercícios: Dinâmica}}}
\date{}


%% Vamos numerar os exercícios usando um contador
\newcounter{exerc}
\setcounter{exerc}{0}


% Para novos ambientes: 
% outros espaços verticais possíveis
% \smallskip \medskip \bigskip
\newenvironment{questão}{%
\refstepcounter{exerc}%
\textbf {Questão \theexerc:}}{\medskip}

%% Exemplo de novo ambiente:
%%
%% \newenvironment{xxx}{aaa}{bbb}
%% 
%% \begin{xxx}
%% aqui e agora
%% \end{xxx}
%% 
%% aaa
%% aqui e agora
%% bbb
%% 
\begin{document}



\maketitle

Entregar até terça-feira. A questão \ref{vale-mais} vale mais pontos!

\begin{questão}
Um corpo com massa de 5 kg é submetido a uma força de intensidade 25N. Qual é a aceleração que ele adquire?
\end{questão}

\begin{questão}\label{vale-mais}
Quando a resultante das forças que atuam sobre um corpo é 10N, sua aceleração é 4 m/s2. Se a resultante das forças fosse 12,5N a aceleração seria de:
\end{questão}

\begin{enumerate}[a)]

\item 2,5 m/s2

\item 5,0 m/s2

\item 7,5 m/s2

\item 2,0 m/s2

\item 12,o m/s2

\end{enumerate}



\begin{questão}
A primeira Lei de Newton afirma que, se a soma de todas as forças atuando sobre o corpo for zero, o corpo \ldots

\begin{enumerate}[a)]

\item terá um movimento uniformemente variado

\item apresentará velocidade constante

\item apresentará velocidade constante em módulo, mas sua direção poderá ser alterada.

\item será desacelerado

\item apresentará um movimento circular uniforme.

\end{enumerate}
\end{questão}



\begin{questão}
Um corpo com massa de 60 kg está na superfície do planeta Marte, onde a aceleração da gravidade é 3,71 m/s2 . De acordo com esses dados, responda:

\begin{enumerate}[a)]
\item Qual é o peso desse corpo na superfície de Marte?
\item Suponha que esse mesmo objeto seja trazido para a Terra, onde g = 9,78 m/s2, qual será o seu peso?


\end{enumerate}
\end{questão}

\end{document}








