\documentclass[extrafontsizes,14pt]{memoir}
\usepackage[utf8]{inputenc}
\usepackage[T1]{fontenc}
\usepackage[brazil]{babel}

% Para citar sites, links  e emails
\usepackage[hyphens]{url} % ou hyperref


\title{50 coisas para se fazer em São Paulo}
\date {}

\begin{document}

% Para forçar fonte maior
%
%
\fontsize{20pt}{24pt}\selectfont

\maketitle


Algumas linhas de texto
para a gente poder ver o
efeito que dá alterar
os parâmetros do tamanho
da fonte do texto. Algumas linhas de texto
para a gente poder ver o
efeito que dá alterar
os parâmetros do tamanho
da fonte do texto. Algumas linhas de texto
para a gente poder ver o
efeito que dá alterar
os parâmetros do tamanho
da fonte do texto.





\begin{enumerate}

\item Fazer uma selfie no Beco do Batman

\item Hip-hop, vinil e tatuagem na Galeria do Rock

\item Empinar pipa no Pipódromo do Parque Ecológico do Tietê
 
\item Descobrir novos sabores de frutas no varejão do CEAGESP

\item Entrar na casa de vidro da Lina Bo Bardi

\item Conhecer a história do seu sobrenome no Museu da Imigração

\item Ouvir canto gregoriano no Mosteiro São Bento

\item Sentir os aromas dos temperos na zona cerealista

\item Caminhar pelo jardim da Fundação Maria Luisa e Oscar Americano

\item Fim de tarde num barco na represa do  Guarapiranga

\item Viajar na história preservada no Museu da Arqueologia e Etnologia

\item Aprender diferentes formas de degustar café no Coffee Lab

\item Cinema ao ar livre na Cinemateca Brasileira

\item Ouvir um concerto na Sala São Paulo

\item Comer nos food trucks do Calçadão Urbanoide

\item Fazer uma aposta no Joquey de São Paulo

\item Aprender se divertindo no Museu Catavento
 
\item Praticar o Maha Mantra nos festivais Hare Krishna do Vrinda
 
\item Explorar as trilhas do Parque Burle Marx

\item Resgatar a história da ditadura na Pinacoteca

\item Ver a fabricação das balas na Rock Candy
 
\item Acompanhar a programação cultural do SESC
 
\item Observar o Copam do alto do Terraço Itália
 
\item Brincar com palavras no Museu da Língua Portuguesa

\item Assistir corrida no autódromo de Interlagos

\item Caminhar entre as diferentes tribos na madruga do Baixo Augusta

\item Comer doces portugueses na Casa Mathilde

\item Pedalar nas ciclovias aos domingos

\item Comprar ferramentas de jardinagem na rua    Florêncio de Abreu e eletrônicos na Santa Ifigênia

\item Assistir a uma ópera no Theatro Municipal

\item Observar o afresco da Capela do Morumbi com anjos de fisionomia indígena

\item Fazer piquenique no Solo Sagrado de Guarapiranga

\item Meditar na lua cheia no Parque Ibirapuera

\item Experimentar pratos típicos da Bolívia na Feira da Kantuta

\item Feirinha dos produtores orgânicos no Parque da Água Branca

\item Andar sob as luzes coloridas da rua Avanhandava

\item Ver a cidade do Pico do Jaraguá

\item Sentir o sabor italiano-paulistano das cantinas do Bixiga

\item Comer canoli no estádio do Juventus

\item Conhecer a história de SP no Pátio do Colégio

\item Andar sem rumo pelo centro velho

\item Se ligar nos tributos no reloginho do Impostômetro da Rua Boa Vista

\item Descobrir um novo restaurante japones na Liberdade

\item Morder o gigantesco pastel do Mercadão

\item Viver a noite egípcia da Casa de Chá Khan el Khalili

\item Caminhar entre Monet, Van Gogh e Portinari no MASP

\item Passear na Maria Fumaça Mooca-Brás

\item Andar entre túmulos e gatos no Cemitério da Consolação

\item Deitar no gramado da pracinha do Pôr-do-Sol enquanto o céu dá um show

\item Ser cobaia da eterna disputa: qual é a melhor pizza de São Paulo?

\end{enumerate}

% Use \vspace*{\fill}
% para empurrar até o fim
% da página.
\vspace*{3cm} 

Blog Visite São Paulo\\

{\small % fonte menor
\url{http://visitesaopaulo.com/blog/index.php/2015/09/marcelo-tas-posta-top-50-coisas-pra-voce-nao-deixar-de-fazer-em-sao-paulo/}}






\end{document}








