\documentclass[12pt, article, oneside]{memoir}
	
	\usepackage [utf8]{inputenc}
	\usepackage[T1]{fontenc}
	\usepackage[brazil]{babel}
	
	% A ideia é fazer um trabalho só, que abranja todos assuntos do segundo videoaula
	\begin{document}
		
		
		\title{ Pequeno texto sobre travessão e aspas}
		\maketitle
		
		% Vamos separar por páginas cada exemplo
		\tableofcontents
		\newpage
		\chapter{Introdução}	

		
			
			
		 Travessão (--)\\
		O travessão é um é um traço maior que o hifen e costuma ser empregado:
		
		--- No discurso direto para indicar a fala da personagem ou a mudança de interlocutor nos diálogos
		\section{Primeiro exemplo}
		
		Exemplo:
		
		--- O que é isso mãe?\\
		--- É o seu presente de aniversário, minha filha\\
		--- Para separar expressões ou frases explicativas, intercaladas.\\
		\chapter{Segundo exemplo}
		   Exemplo:
		   
		   `` E logo me apresentou a mulher, --- uma estimável senhora --- e à filha''(Machado de Assis)\\
		   --- Para destacar algum elemento no interior da frase, servindo muitas vezes para realçar o aposto.\\
		   \newpage
		  \chapter{Terceiro exemplo}
		   Exemplo:
		   
		   `` Junto do leito meus poetas dormem\\
		   O Dante, a bíblia Shakespeare e byron--- \newline
		   Na mesa confundidos.''(Álvares de Azevedo).
		   \newpage
		   \chapter{Introdução sobre aspas}
		  	\item Aspas(`` '')\\
		  	AS aspas têm como função destacar uma parte do texto. São empregadas:\\
		  	--- Antes e depois de citações ou transquiçõses textuais.
		  	\newpage
		  	\chapter{Primeiro exemplo}
		  	Por exemplo:\\
		  	como disse Machado de Assis: ``A melhor definição do amor não vale um beijo de moça namorada.''
		  	
		  	--- Para separar nome de livros ou legendas\\
		  	\newpage
		  	\chapter{Segundo exemplo}
		  	Exemplo:
		  	
		  	Camões escreveu ``Os Lusíadas'' no século XVI.
		  
		  	
		  	
		   
		   
		
		
		
		
		
		
		
				
	
	
	\end{document}