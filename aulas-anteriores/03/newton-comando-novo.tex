% Darnerson Pereira de Sousa
% Modificado 2016-03-14
%
% Mudanças
% 15-03-16 - remover tabulaçoes, 
\documentclass[12pt]{memoir}
\usepackage[utf8]{inputenc}
\usepackage[T1]{fontenc}
\usepackage[brazil]{babel}

\newcommand{\IsaacNewton}{I.N.}
\begin{document}
\begin{flushleft}

  \small  Um texto sobre Isaac Newton utilizando \LaTeX\\
  Terceira vídeo aula
  
\end{flushleft}
\begin{center}
  \LARGE

  \textbf{Um físico chamado Isaac Newton}
\end{center}  

\IsaacNewton\ foi um personagem muito importante na história da ciência, principalmente na área da \emph{físca} e da \emph{matemática}. Nascido em 1643, mesmo ano da morte de Galileu Galilei, em uma cidade localizada na inglaterra, \IsaacNewton\ foi um gênio da sua época, além de \emph{físca} e \emph{matemática} ele estudou filosofia, astronomia, química, teologia, astrologia, entre outras ciências, ele, juntamente  com vários outros cientistas e pensadores da época, acreditavam que o estudo dessas ciências possibilitaria a compreensão e estudos dos fenômenos naturais.

\IsaacNewton\ ficou muito conhecido por todos os trabalhos, pesquisas e investigações experimentais que realizou. As investigações experimentais eram cheias de vigor \emph{matemática}, e se tornaram um verdadeiro modelo de investigação para as ciências do séculos posteriores.\textbf{
  \textsc{Dentre os muitos trabalhos que \IsaacNewton\ elaborou, podemos citar:}}


\begin{itemize}
\item desenvolvimento da série de potência de um binômio, que hoje é conhecido pelo nome de \emph {binômio de Newton}; 
  
\item A criação e desenvolvimento do \emph{cálculo diferencial} e \emph{cálculo integral}, o que é uma ferramenta muito importante para o estudo dos fenômenos físicos. Além de ser ele o criador dessa ferramenta, foi ele também quem a utilizou pela primeira vez;
\item O estudo sobre fenômenos óticos que possibilitaram a elaboração teórica sobre a cor dos corpos;
\item O estudo das leis dos movimentos, lançando as bases da mecânica;
\item O desenvolvimento das primeiras ideias sobre a \emph {gravitação universal}.  
\end{itemize}

\begin{flushright}
  Por  Marcos Aurélio da Silva
\end{flushright}

\begin{center}
  \begin{Huge}
    \textbf{Fim}  
  \end{Huge}
\end{center}
\end{document}  
