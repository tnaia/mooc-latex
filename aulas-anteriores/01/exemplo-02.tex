% Michelle Senna
% Modificado 2016-03-01
\documentclass[12pt]{memoir}

\usepackage[utf8]{inputenc}
\usepackage[T1]{fontenc}
\usepackage[brazil]{babel}

\begin{document}


\title {O que é o \LaTeX}
\maketitle 

Trata-se de um conjunto de macros ou marcações para o processador de textos TeX.

É utilizado amplamente pela comunidade científica e matemática mundial
devido à sua grande qualidade tipográfica, entre outros. O LaTeX
(lê-se latec) fornece um conjunto de macros alto-nível que torna mais
fácil e rápida a produção de documentos em TeX e é utilizado para
produzir todo o tipo de documentos como por exemplo livros, relatórios
e artigos.

O objetivo do LaTeX é que o autor possa se distanciar da apresentação
visual do trabalho e assim concentrar-se no seu conteúdo. Possui
formas de lidar com bibliografias, citações, formatos de páginas,
referências e tudo mais que não seja relacionado com conteúdo do
documento em si.

O autor depara-se com um paradigma em que em vez de lidar com
conceitos visuais lida com conceitos lógicos, e consequentemente mais
independentes da apresentação, como por exemplo Capítulos,
Secções, Resumos, Partes e tabelas, permitindo no entanto que haja
liberdade para que o utilizador, se assim o desejar, possa declarar o
seu próprio esquema de formatação.

Ao contrário dos sistemas WYSIWYG, o conceito do LaTeX é visualizar a
apresentação do resultado final do texto apenas após a preparação do
mesmo, evitando assim distrações quanto à forma durante a preparação
do conteúdo.

Como foi dito, o LaTeX não é algo imutável, e como tal suporta formas
de estilizar e formatar os documentos como bem entendermos. Para isso
serve-se de classes e pacotes que determinam o estilo e a formatação
do documento bem como outras funcionalidades especificas. Os muitos
pacotes criados para o LaTeX são essenciais para que os utilizadores
do sistema tenham maior liberdade na criação dos documentos. Muitos
pacotes nem sempre adicionam novas funcionalidades, mas modificam o
tratamento padrão dado a certas funções, criando mais diversidade de
apresentação visual para o universo dos documentos produzidos em
LaTeX. Os pacotes podem ser obtidos através da CTAN.







\end {document}
