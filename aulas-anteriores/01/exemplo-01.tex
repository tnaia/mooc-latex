% Darnerson Pereira de Sousa
% Modificado 2016-03-01
\documentclass[12pt]{memoir}
\usepackage[utf8]{inputenc}
\usepackage[T1]{fontenc}
\usepackage[brazil]{babel}


\begin{document}
	
\title{	Breve texto sobre \TeX\ e \LaTeX}
\maketitle

\begin{itemize}	
\item \TeX \TeX é um sistema de editor de textos criado por
  Donald. Knuth para produção de material (Livros, artigos, etc.) de
  alta qualidade tipográfica. \TeX é de fato um processador de macros
  e possui poderosa capacidade de programação. junto com o sistema
  está disponível um pequeno conjunto de macros denominadas "plain
  \texciicircum". Quando alguém diz que está escrevendo um texto
  diretamente em \TeX , geralmente, quer dizer que está usando este
  conjunto mínimo de macros "plain \TeX ".

\item \LaTeX \LaTeX é um conjunto de macros \TeX originalmente escrito
  por Leslie Lamport que implementam um sistema de preparação de
  documentos. \LaTeX define uma linguagem de "marKup" do mais alto
  nível permitindo descrever o documento em termos de sua estrutura
  lógica e não apenas do seu aspecto visual. Usando diferentes classes
  de documentos e macros ("packages") adicionais ousuário pode
  produzir uma grande variedade de "layouts". Sua primeira versão,
  largamente usada foi a 2009, lançada em 1985.


\end{itemize}	
	
	
	
	
	
	
	
	
	
\end{document}
