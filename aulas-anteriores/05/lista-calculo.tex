%Projeto aula 5: Matemática no LaTeX

\documentclass[12pt]{memoir}
\usepackage[utf8]{inputenc}
\usepackage[T1]{fontenc}
\usepackage[brazil]{babel}

\usepackage{amsmath}
\usepackage{amssymb}

\newcounter{exerc}

% diferentes espaçamentos verticais
% \smallskip, \medskip, \bigskip

\newenvironment{exercício}{\stepcounter{exerc}
\par{\textbf{Exercício.}
\theexerc\par\smallskip}}{\par\bigskip}
 
\begin{document}
\title{\textbf{Lista de Cálculo \\ Licenciatura em Física}}
\date{}
\maketitle

\begin{exercício}
Calcule a área da região determinada por todos os pontos~(\(x,y\))~do plano tais que~\(x\geq 1+ y^2\)~ e ~\(2x+y\leq 2\)
\end{exercício}

\begin{exercício}
Calcule a área entre cada parábola e o eixo~\(x.\)
\begin{enumerate} [a)]
\item \(x^2+3y = 9\)
\item \(3x^2+4y = 48\)
\item \(x^2+4x+2y = 0\)
\item \(y+x^2-1 = 0\)
\end{enumerate}
\end{exercício}

\begin{exercício}
Ache a área limitada pela curva dada, o eixo \(x\) e as retas verticais dadas:
\begin{enumerate}[a)]
\item \(y = \sqrt{x+2},\)~~~~~~~~~~\(x = 2\)~~~e~~~\(x = 7\)
\item \(y = \sqrt[3]{3-x},\)~~~~~~~~~~\(x = -5\)~~~e~~~\(x = 3\)\item \(y =x\sqrt{5-x^2},\)~~~~~~~~~~\(x = 0\)~~~e~~~\(x = \sqrt{5}\)
\item \(y = x^2+4x+4,\)~~~~~\(x = -1\)~~~e~~~\(x = 1\)
\end{enumerate}
\end{exercício}
\end{document}