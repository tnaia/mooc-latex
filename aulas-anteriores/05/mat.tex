\documentclass[12pt]{memoir}
\usepackage[utf8]{inputenc}
\usepackage[T1]{fontenc}
\usepackage[brazil]{babel}
\usepackage{amsmath}
\usepackage{amssymb}
\usepackage{amsthm}

\begin{document}
\title{Lista de exercícios de matemática }
\maketitle	

1) Quais das equações abaixo são do 2º grau?\par


()\(x - 5x + 6 = 0\)\par   


             
()\(x^2 - 7x + 10 = 0\)\par
()\(0x^2 + 4x - 3 = 0\)\par
 ()\(2x^3 - 8x^2 - 2 = 0\)\par
 
 	
	2) Resolva as inequações:\par
	a)\(x + 1 < 0\)\par
	b) \(x^2 - 25 >0\)\par
	c) \(3x - 2 \leq 0\)\par
	d) \(2x + 4 \geq 0\)\par
	
	3) Resolva as equaçoes seguintes:\par
	
	a) \(x^3 - 8 = 0\)\par
	b) \(x + \sqrt9 = 10\)\par
	d) \(-3 + \sqrt{x + 1} = 0\)\par	
	e) \(-5 +x = \sqrt[5]{x+1}\)\par
	
	4) Um sexto de uma pizza custa 3 reais, quanto custa:\par
	\begin{enumerate}[a)]
	\item \(\frac{3}{6}\)
		
	\item \(\frac{5}{6}\)
	
	\item \(\frac{2}{3}\) ou \(\displaystyle\frac{2}{3}\)	 ou \(2/3\)
	\end{enumerate}
	
	5) Encontre o valor numérico das expressões, sabendo que \[(a + b) = 7\ e\ (a - b) = 1\]\par
	
	a) \(3+\frac{a+b}{5}\) \par
	b)\(6 - 2.3 +\frac{9}{5 + b}\)\par
	c) \(a + b - \frac{1}{b-a} + 2^3\)\par
	
	6) Encontre o valor de x  em cada caso:\par
	
	a)\(\log{2}x - 1 = \log{3}x+1\)\par
	
	b) \(x + \sin x + 4 = 3\sin 2x\)\par
	
	c) \(log_3 {x} + 5 = 8\)\par
	
	d) \(3x + \ln x +1 = 2\ln x\)\par
	
	7) Calcule:
	
	   \[
	   lim_{x\to1}\frac{(x-1)(x + 2)}{x-1}	
	   \]
	
	
	
	
	
	
	
% Outro modo de fazer.
%	
%\begin{enumerate}[1)]
%\item Primeira questão
%  \begin{enumerate}[a)]
%  \item item a
%  \item item b
%  \item item c
%  \end{enumerate}
%\item Segunda questão
%\item Terceira questão
%\item Quarta questão
%\end{enumerate}
	
	
\end{document}
