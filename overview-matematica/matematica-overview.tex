\documentclass[article,openany]{memoir}

% Codificação da entrada e da saída ==================================
\usepackage[utf8]{inputenc}
\usepackage[T1]{fontenc}

% Idioma =============================================================
\usepackage[brazil]{babel}

% Hiperlinks =========================================================
\usepackage{url}

% Tipografia =========================================================
% (Podem ser removidos sem problema.)
\usepackage{lmodern}
\usepackage{microtype}

% Autor e título =====================================================
\title{Bib\TeX: tipos de entradas e campos}
\author{Tássio Naia dos Santos}

% Abreviações ========================================================
\newcommand{\obrigatorios}{%
   \par Obrigatório\dotfill\ignorespaces}

\newcommand{\opcionais}{
   \par Opcional\dotfill\ignorespaces}


% Deixando a coluna de texto um pouco mais gorda =====================
\settypeblocksize{23cm}{13cm}{*}
\setulmargins{*}{*}{1}
\setlrmargins{*}{*}{1}
\checkandfixthelayout

\begin{document} % ===================================================


%\maketitle

\noindent{\LARGE Matemática, ao \(\infty\), e além!}\hfill 03 de julho de 2014

\medskip

\noindent Tássio Naia dos Santos

\medskip

\noindent 25/07/2014: correções gramaticais (Shayenne Moura).

\bigskip

\begin{abstract}
Apresentamos algumas construções matemáticas comuns: sobrescrito,
subscrito, fórmulas na linha e em destaque, letras gregas, subscrito
em limites e somatórios, integral. 

Ficam faltando: Vetores, acentos, equações multilinha, cases,
matrizes, ambientes para teorema e companhia.
\end{abstract}

\chapter{Últimas alterações}
\chapter{Modos de escrita matemática}

Para escrever uma expressão \emph{em linha} no modo matemático basta
escrevê-la entre ``\verb/\(/`` e~``\verb/\)/''. Por exemplo,
``\verb/\(x + 3y = \infty\)/'' produz~``\(x + 3y = \infty\)''. Os operadores para
demarcar uma expressão matemática em destaque são~``\verb/\[/''
e~``\verb/\]/''. Exemplo: ``\verb/\[a + b + c + \cdots = 20\]/'' gera

\[a+b+c+\cdots = 20\]

Note que os espaços são ignorados : ``\verb/\(a   b c\)/''
resulta em~\(a   b c\).

Reticências. Há mais de um modo de indicar omissão de pedaços de uma
expressão matemática. Temos a disposição vários tipos de reticências:
\begin{itemize}
\item ``\verb/\(a + b + \cdots + z\)/'' produz \(a+b+\cdots+z\), 
\item ``\verb/\(\langle a,b,c,\ldots,z \rangle\)/'' produz~\(\langle
  a,b,c,\ldots,z\rangle\).\footnote{Um modo de ler
    \cs{langle}-\cs{rangle} é: ângulo esquerdo (\emph{left angle}), e
    ângulo direito \emph{right angle}, respectivamente.}
\end{itemize}

\chapter{Subscrito e sobrescrito}

Usamos o operador~``\verb/^/'' (acento circunflexo\footnote{Geralmente
  é possível obter esse caractere pressionando a tecla com o caractere
  circunflexo e então pressionando a barra de espaço.}) para colocar o
elemento seguinte em sobrescrito: ``\verb/\(a^b+c\)/'' produz~\(a^b +
c\). O operador para subscrito é o underline\footnote{O underline é
  também chamado subtraço, underscore ou ou traço inferior.}):
``\verb/a_b + c/'' produz~\(a_b + c\).

Tente escrever:
\begin{itemize}
\item \(a_1 + a_2 + \cdots + a_n = 30\)
\item \(a^{b + c + d + \cdots} = 10^{200}\)
\end{itemize}

\chapter{Agrupamentos}

Assim como agrupamos palavras quando usamos comandos como \cs{emph},
podemos agrupar expressões. Veja a diferença entre~\(a^{b + c}\)
e~\(a^b + c\). Como você faria para obter cada um desses
efeitos? (Resposta no rodapé.\footnote{Resposta (não espie antes de tentar!) a resposta
  é~\texttt{\char`\\(a\char`\^\char`{ b + c\char`}\char`\\)}
e~\texttt{\char`\\(a\char`\^b + c\char`\\)}})

\chapter{Frações e binomial}

Podemos indicar uma fração usando a barra: ``\verb-\(a/b\)-'' produz
``\(a/b\)''. Outro modo é usando o comando \cs{frac}. Esse comando
recebe dois argumentos: o primeiro é o numerador da fração, e o
segundo é o denominador. Por exemplo:~``\verb/\(\frac{a}{b}\)/''
produz~\(\frac{a}{b}\), e em modo matemático de destaque obtemos
\[
\frac{a}{b}
\]

Para escrever um binomial, usamos um comando um pouco diferente,
\cs{choose}. Ele age \emph{dentro} de um grupo:
``\verb/\(a + { n \choose k } + b\)/'' produz \(a + { n \choose k } +
b\). (Lembramos que, em geral, grupos são partes do texto delimitadas
por chaves.)

\chapter{Alguns símbolos}

Não há limite para a quantidade de símbolos usados em textos matemáticos.
Para descobrir como produzir um símbolo: você pode consultar os seguintes textos, disponíveis gratuitamente online.\footnote{Se está em um computador com \LaTeX\ instalado, pode acessar esses documentos usando os comandos ``\texttt{texdoc lshort}'' e~``\texttt{texdoc comprehensive}''.}
\begin{itemize}
\item Uma não tão curta introdução a \LaTeX
\item The Comprehensiive \LaTeX\ symbol list (A lista abrangente de símbolos do \LaTeX)
\end{itemize}

\section{Letras gregas}

Letras gregas (\(\alpha,\beta,\gamma,\delta,\Delta,\lambda,\ldots\),
veja tabela~\ref{tab:greek}).

\begin{table}[h]
\caption{Letras gregas (modo matemático).}\label{tab:greek}
\begin{tabular}{llllllll}
$\alpha$       & \cs{alpha}       & $\theta$     & \cs{theta}     & $o$          & \cmdprint{o}   & $\upsilon$ & \cs{upsilon} \\
$\beta$        & \cs{beta}        & $\vartheta$  & \cs{vartheta}  & $\pi$        & \cs{pi}        & $\phi$     & \cs{phi}     \\
$\gamma$       & \cs{gamma}       & $\iota$      & \cs{iota}      & $\varpi$     & \cs{varpi}     & $\varphi$  & \cs{varphi}  \\
$\delta$       & \cs{delta}       & $\kappa$     & \cs{kappa}     & $\rho$       & \cs{rho}       & $\chi$     & \cs{chi}     \\
$\epsilon$     & \cs{epsilon}     & $\lambda$    & \cs{lambda}    & $\varrho$    & \cs{varrho}    & $\psi$     & \cs{psi}     \\
$\varepsilon$  & \cs{varepsilon}  & $\mu$        & \cs{mu}        & $\sigma$     & \cs{sigma}     & $\omega$   & \cs{omega}   \\
$\zeta$        & \cs{zeta}        & $\nu$        & \cs{nu}        & $\varsigma$  & \cs{varsigma}  & \\
$\eta$         & \cs{eta}         & $\xi$        & \cs{xi}        & $\tau$       & \cs{tau}       & \\
$\Gamma$       & \cs{Gamma}       & $\Lambda$    & \cs{Lambda}    & $\Sigma$     & \cs{Sigma}     & $\Psi$     & \cs{Psi}     \\
$\Delta$       & \cs{Delta}       & $\Xi$        & \cs{Xi}        & $\Upsilon$   & \cs{Upsilon}   & $\Omega$   & \cs{Omega}   \\
$\Theta$       & \cs{Theta}       & $\Pi$        & \cs{Pi}        & $\Phi$       & \cs{Phi}       
\end{tabular}
\end{table}


\section{Operadores ``relacionais''}

\begin{tabular}{llllllll}
$=$ & \verb/=/ &
$\neq$ & \cs{neq} &
$<$ & \verb/</ &
$>$ & \verb/>/ \\
$\leq$ & \cs{leq} &
$\geq$ & \cs{geq} &
$\in$ & \cs{in} &
$\notin$ & \cs{notin} \\
$\subset$ & \cs{subset} &
$\supset$ & \cs{supset} &
$\subseteq$ & \cs{subseteq} &
$\supseteq$ & \cs{supseteq}
\end{tabular}

\section{Soma, limite, integral}

Os símbolos \cs{sum} (somatório), \cs{lim} (limite) e \cs{int}
(integral) são em geral grandes, e têm um comportamento diferente
quando usamos subscrito e sobrescrito.

\begin{itemize}
\item ``\verb/\(\sum_i a_i\)/'' produz \(\sum_i a_i\)
\item ``\verb/\(\sum_{j = 0} ^ n b_j\)/'' produz \(\sum_{j = 0} ^ n b_j\)
\item ``\verb/\(\lim_{x \to c} f(x) = f(c) \pm K\)/'' produz \(\lim_{x \to c} f(x) = f(c) \pm K\)
\item ``\verb/\(\int_a ^ b x dx\)/'' produz \(\int_a ^ b x dx\)
\end{itemize}

\section{Operadores}

\begin{tabular}{llllll}
\(\cdot\) & \cs{cdot} &
\(\times\) & \cs{times} &
\(\circ\) & \cs{circ} \\
\(\land\) & \cs{land} &
\(\lor\) & \cs{lor} &
\(\lnot\) & \cs{lnot} \\
\(\pm\) & \cs{pm} &
\(\mp\) & \cs{mp} &
\end{tabular}

\chapter{Exercícios}

Experimente produzir as seguintes expressões matemáticas, ou, melhor
ainda, encontre expressões matemáticas reais desse tipo para
experimentar! Se necessário, consulte o texto ``lshort'': Uma não tão
curta introdução ao \LaTeX, disponível gratuitamente, e fácil de
encontrar na internet.
\begin{enumerate}
\item \(\displaystyle a + {3x \choose x} = \beta^2\)
\item \(\displaystyle E=mc^2\)
\item \(\displaystyle \sum_{i=0}^\infty 0 = 0\)
\item \(\displaystyle \lim_{x\to \infty} \frac{1}{x} \leq  0 + \varepsilon \)
\item \(\displaystyle \sin^2 x + \cos^2 x = 1\)
\item \(\displaystyle \Delta + \frac{\partial f}{\partial x} \geq  h(y)\)
\item \(\displaystyle a \in A \subset B\)
\end{enumerate}

\end{document}
