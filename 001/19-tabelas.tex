% Tabelas

% Cabeçalho para artigo

\documentclass[12pt,article,a5paper]{memoir}

% Escrevendo em português e utf8
\usepackage[utf8]{inputenc}
\usepackage[T1]{fontenc}
\usepackage[brazil]{babel}
\usepackage{hyperref}

\begin{document}

\listoftables
Veja a Tabela~\ref{tab:horarios}.
\begin{table}
% Horários na semana
%       | seg | ter | qua |...
% 10:00 |     |     |     |
% 11:00 |     |     |     |
% 13:00 |     |     |     |
% 14:00 |     |     |     |
\begin{center}
\begin{tabular}{rlllll}
\hline
   & seg & ter & qua & qui & sex \\
   \hline
9 & João& José&     &     &     \\
11 &     &     &     &     &     \\
13 &     &     & Tina&     &     \\
14 &     &     &     &     &     \\
\hline
\end{tabular}
\end{center}
\caption{Horários da semana}
\label{tab:horarios}
\end{table}

\clearpage





\begin{tabular}{c@{ está valendo $\to$ }r@{,}l}
maçã   &   2 & 30 \\
banana &  10 & 50 \\
pera   & 999 & 00 
\end{tabular}




\clearpage
% Currículo
% período |atividades...
%
\begin{tabular}{r|p{8cm}}
2011--hoje & estou trabalhando com arrumação da minha casa (papéis, papéis\ldots)\\
2005--2010 & graduação metafísica\newline
Universidade Dali\\[10pt]
2005--2010 & graduação filosofia\\
2005--2010 & graduação matemática\\
\end{tabular}


\cleartable

% Existem programas para ajudar a formatar
% tabelas (online, o próprio texmaker, etc.)

% Pacotes úteis
% - Tabelas grandes: longtable
% - Tabelas mais sofisticadas: booktabs




\end{document}