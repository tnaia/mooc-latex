\documentclass[article,openany]{memoir}

% Codificação da entrada e da saída ==================================
\usepackage[utf8]{inputenc}
\usepackage[T1]{fontenc}

% Idioma =============================================================
\usepackage[brazil]{babel}

% Hiperlinks =========================================================
\usepackage{url}

% comando \text para texto em fórmulas ===============================
\usepackage{amsmath}

% Tipografia =========================================================
% (Podem ser removidos sem problema.)
\usepackage{lmodern}
\usepackage{microtype}

% Autor e título =====================================================
\title{Matemática: miscelânea}
\author{Tássio Naia dos Santos}

% Abreviações ========================================================
\newcommand{\obrigatorios}{%
   \par Obrigatório\dotfill\ignorespaces}

\newcommand{\opcionais}{
   \par Opcional\dotfill\ignorespaces}

\newcommand*{\extensao}[1]{\texttt{#1}}

\newcommand*{\pacote}[1]{\textsf{#1}}

\newcommand*{\ambiente}[1]{\textsf{#1}}

\newcommand*{\astt}[1]{% <--- aspas simples com texto monoespaçado
  `\thinspace\texttt{#1}\thinspace'} 

\newcommand*{\as}[1]{% <--- aspas simples com texto normal
  `\thinspace#1\thinspace'}

\newenvironment{exemplo}{\begin{center}}{\end{center}}
% Deixando a coluna de texto um pouco mais gorda =====================
\settypeblocksize{23cm}{13cm}{*}
\setulmargins{*}{*}{1}
\setlrmargins{*}{*}{1}
\checkandfixthelayout

\begin{document} % ===================================================


%\maketitle

\noindent{\LARGE Matemática: miscelânea}\hfill 28 de julho de 2014

\medskip

\noindent Tássio Naia dos Santos

% últimas modificações ===============================================
%\medskip
%\noindent 25/07/2014: correções gramaticais (Shayenne Luz Moura).

\bigskip

\begin{abstract}
Algumas construções comuns. Texto em fórmulas, ambiente
\ambiente{cases}, derivadas, integrais (simples, duplas, em curvas
fechadas e parciais), entre outros.
\end{abstract}

\chapter{Texto em fórmulas e ambiente \ambiente{cases}}

O pacote \pacote{amsmath}, da Sociedade Matemática Americana (American
Mathemaical Society) habilita muitos comandos úteis para escrever
expressões matemáticas. Entre eles está o comando \cs{text}, que
permite inserir texto ``normal'' dentro de fórmulas matemáticas. Por exemplo,
\begin{exemplo}
 \verb/a_{\text{médio}}/\quad produz\quad \(\displaystyle a_{\text{médio}}\).
\end{exemplo}
Outro acréscimo desse pacote é o ambiente \ambiente{cases}, que permite a seguinte construção
\begin{verbatim}
\[
 f(x) = 
 \begin{cases}
   2 & \text{se \(x\) é um múltiplo inteiro de \(\pi\)} \\
   1 & \text{se \(x\) é um número racional}             \\
   0 & \text{caso contrário}
 \end{cases}
\]
\end{verbatim}
e que produz
\[
f(x) = 
 \begin{cases}
   2 & \text{se \(x\) é um múltiplo inteiro de \(\pi\)} \\
   1 & \text{se \(x\) é um número racional} \\
   0 & \text{caso contrário}
 \end{cases}
\]

%% \chapter{Integrais}
%% 
%% No tex.stackexchange parece unanimidade que o jeito certo é:
%% \begin{verbatim}
%% \newcommand*\diff{\mathop{}\!\mathrm{d}}
%% \newcommand*\Diff[1]{\mathop{}\!\mathrm{d^#1}}
%% \end{verbatim}
%% 
%% Algumas vezes é preciso fazer alguns ajustes finos no modo como certas
%% expressões são formatadas. Integrais são um exemplo disso. Se
%% simplesmente escrevemos os símbolos de uma integral, por exemplo
%% ``\verb/\int_0^\infty x dx/'', obtemos
%% \[\int_0^\infty x dx.\]
%% Dois ajustes são geralmente feitos nessa expressão. Primeiro, note que
%% o ``d'' no diferencial~\(dx\) não é uma variável, isto é, não
%% representa um número ou parâmetro de uma função. Para ressaltar esse
%% fato, em geral empregamos outra fonte matemática para essa letra,
%% escrevendo ``\verb/\mathrm{d}x/'' em vez de ``\verb/dx/''. O resultado é
%% \[\int_0^\infty x \mathrm{d}x.\]
%% O outro ajuste que fazemos é no espaçamento. Há espaço \emph{demais}
%% entre o símbolo~\(\int\) da integral e a expressão integrada
%% ``\(x\)''. Por outro lado, não existe espaço algum antes do
%% diferencial ``\(\mathrm{d}x\)''. Resolvemos isso escrevendo, por exemplo
%% ``\verb/\int^\infty\!\!\! x \,\mathrm{d}x/''.
%% \[
%% \int_0^\infty\! x \,\mathrm{d}x.
%% \]

 

\end{document}
