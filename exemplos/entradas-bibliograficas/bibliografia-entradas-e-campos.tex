\documentclass[article,openany]{memoir}

% Codificação da entrada e da saída ==================================
\usepackage[utf8]{inputenc}
\usepackage[T1]{fontenc}

% Idioma =============================================================
\usepackage[brazil]{babel}

% Hiperlinks =========================================================
\usepackage{url}

% Tipografia =========================================================
% (Podem ser removidos sem problema.)
\usepackage{lmodern}
\usepackage{microtype}

% Autor e título =====================================================
\title{Bib\TeX: tipos de entradas e campos}
\author{Tássio Naia dos Santos}

% Abreviações ========================================================
\newcommand{\obrigatorios}{%
   \par Obrigatório\dotfill\ignorespaces}

\newcommand{\opcionais}{
   \par Opcional\dotfill\ignorespaces}


% Deixando a coluna de texto um pouco mais gorda =====================
\settypeblocksize{23cm}{13cm}{*}
\setulmargins{*}{*}{1}
\setlrmargins{*}{*}{1}
\checkandfixthelayout

\begin{document} % ===================================================


%\maketitle

\noindent{\LARGE Bib\TeX: tipos de entradas e campos}\hfill 03 de junho de 2014

\medskip

\noindent Tássio Naia dos Santos

\medskip

\noindent 25/07/2014: correções gramaticais (Shayenne Luz Moura).

\bigskip

\begin{abstract}
Apresentamos os tipos de entradas bibliográficas mais comuns
que são compreendidos pelo Bib\TeX\ (livro, artigo, relatório
técnico). Também apresentamos os campos obrigatórios e facultativos de
cada tipo de entrada.
\end{abstract}


\chapter*{Agradecimentos \& crédito}

Agradecemos a Norman Walsh\footnote{\url{http://nwalsh.com/tex/}} e a
Juan Antonio Navarro
Pérez\footnote{\url{http://www.navarroj.com/latex/}} de onde obtivemos
parte deste material. A tradução e os erros são de responsabilidade
de~Tássio Naia dos~Santos.

\chapter{Introdução e nomes dos bois}

Chamamos de \emph{entrada bibliográfica} o conjunto de informações
sobre alguma fonte ou referência bibliográfica. Um~\emph{campo} de uma
referência é uma ``unidade de informação'' sobre a entrada: por
exemplo, o nome de seu autor ou a data da sua~publicação.

As entradas bibligráficas são armazenadas em um ou mais arquivos
bibliográficos (geralmente de extensão~``\texttt{.bib}''). As
informações de uma mesma entrada são agrupadas; marcamos o tipo da entrada
(livro, artigo, tese de doutorado, etc.) e anotamos informações sobre
ela. O tipo de entrada e as informações são usadas pelo
Bib\TeX\ para formatar corretamente a referência.

Um exemplo: Uma referência a um artigo (article), entitulado ``Pé de
Laranja Lima'' (title), escrito por José Pereira (author) em 1930
(year) e publicado pela editora Patatas (publisher) pode ser anotado da
seguinte forma:

\begin{verbatim}
@ARTICLE{jose,
     AUTHOR = {Jos\'e Pereira},
      TITLE = {Pé de Laranja Lima},
  PUBLISHER = {Patatas},
       YEAR = {1930}
}
\end{verbatim}

Onde ``\verb/jose/'' é um rótulo, um codinome ``arbitrário'', que
podemos escolher para usar em nossos textos sempre que citarmos esta
entrada bibliográfica (comando ``\verb/\cite{jose}/'').

Na verdade, os rótulos não podem ser qualquer coisa. Eles \emph{não
podem conter} acentos, cedilha ou vírgula. Você estará em águas
seguras se usar apenas os caracteres abaixo.

\begin{verbatim}
abcdefghijklmnopqrstuvwxyz
ABCDEFGHIJKLMNOPQRSTUVWXYZ
0123456789-_+=:?!*
\end{verbatim}

\chapter{Campos}

As informações que identificam uma referência variam de acordo com o
tipo de entrada bibliográfica. Sendo assim, é natural usar campos
distintos na bibliografia; uma referência a uma tese deve incluir
o nome da escola, mas isso não faz muito sentido para
livros em~geral. Os~campos (informações) podem ser de três tipos:


\begin{description}
\item[Obrigatório]
a omissão deste campo (por exemplo, um livro sem informação sobre
a~editora) desorienta o programa Bib\TeX; produzirá uma mensagem de
erro e pode resultar em uma bibliografia mal-formatada. Se a
informação requerida não for relevante para a entrada que você está
editando, é provável que você esteja usando um tipo de entrada
bibliográfica inadequado (por exemplo, está usando tipo de entrada
\emph{artigo} para uma~\emph{tese}).


\item[Opcional]
a informação do campo será usada se presente, mas pode ser omitida sem
causar problemas de formatação. Uma referência deve conter qualquer
informação disponível que possa ajudar o leitor, então, sempre que
possível, inclua os campos opcionais (aplicáveis) à entrada
bibliográfica.


\item[Ignorado] o campo é ignorado. Por padrão, o Bib\TeX\ ignora qualquer
campo que não seja obrigatório ou opcional, de modo que você pode
incluir quaisquer campos que quiser em uma entrada bibliográfica. É
uma boa ideia colocar toda informação relevante sobre uma referência
na sua entrada bibliográfica. Por exemplo, se você quiser manter o
resumo (abstract) de um artigo em um arquivo no computador, coloque-o
no campo abstract de sua entrada bibliográfica. O banco de referências
bibliográficas é provavelmente um bom lugar para manter o resumo, e
alguns estilos de bibiografia permitem usar essa informação, digamos,
para compor uma bibliografia anotada.
\end{description}

\section{Lista dos campos mais comuns}

\begin{description}
\item[Address]

Endereço da editora. Para grandes editoras, basta fornecer a
cidade. Para pequenas editoras, você pode ajudar o leitor fornecendo o
endereço completo.

\item[Author]

Nome(s) do(s) autor(es), separados por ``\verb/and/''. 

Exemplo: ``\verb/Jorge Amado and S\'ergio Formosa/''. 

Você também pode escrever os nomes no formato sobrenome-vírgula-nome.

Exemplo: ``\verb/Amado, Jorge and Formosa, S\'ergio/''.

\item[Booktitle]

Título do livro citado.

\item[Chapter]

Número do capítulo.

\item[Edition]

A edição de um livro. Por exemplo, ``\verb/segunda/''.

\item[Editor]


Nome(s) do(s) editor(es). Se existe também um campo ``author'', então
o campo ``editor'' fornece o editor do livro ou coleção em que a
referência aparece.

\item[Howpublished]

Como foi publicado. Por exemplo, notas de aula.

\item[Institution]

A instituição que publicou o trabalho.

\item[Journal]

Nome do journal (periódico).

\item[Key]

Usado para controlar a ordem alfabética e para criar um rótulo quando
a entrada não possui campos ``author'' e ``editor''. Este campo não
deve ser confundido com o rótulo (que aparece no início da entrada): o
rótulo é usado para referência quando uma citação é feita no documento
(usando o comando cite).

\item[Month]

O mês em que o trabalho foi publicado ou, caso não tenha sido
publicado, a data em que foi escrito.

\item[Note]

Qualquer informação adicional que possa ajudar o leitor.

\item[Number]

O número do journal (periódico), revista, ou relatório técnico. Uma
edição de um journal (periódico) ou revista é geralmente identificada
pelo seu volume e número; a organização que produz um relatório
técnico geralmente lhe atribui um número.

\item[Organization]

A organização que promove a conferência.

\item[Pages]

Um número de página ou intervalo de páginas como~``\verb/42--111/'';
você pode também ter vários intervalos, separados por
vírgulas~``\verb/7,41,73--97/''.

\item[Publisher]

Nome da editora.

\item[School]

O nome da escola onde a tese foi escrita.

\item[Series]

O nome de uma série ou conjunto de livros. Quando referenciar um livro
inteiro, o campo ``title'' fornece o seu título e o campo opcional
``series'' fornece o nome da série em que o livro foi publicado.

\item[Title]

O título do trabalho.

\item[Type]

O tipo de relatório técnico. Por exemplo, ``Relatório de pesquisa''.

\item[Volume]

O volume do journal (periódico) ou tomo do livro.

\item[Year]

O ano da publicação ou, para um trabalho que não foi publicado, o ano
em que foi escrito. O texto deste campo deve conter apenas dígitos.
\end{description}

\chapter{Entradas}
%Entry types

Quando escrevemos uma entrada de uma referência, a primeira coisa que
precisamos decidir é qual tipo de entrada ela será. Nenhum sistema de
classificação é completo, mas o Bib\TeX\ possui um conjunto de~tipos
de~entradas razoavelmente abrangente.

% When entering a reference in the bibliography database, the first
% thing to decide is what type of entry it is. No fixed classification
% scheme can be complete, but BibTeX provides enough entry types to
% handle almost any reference reasonably well.


O Bib\TeX\ ignora se o tipo da entrada é escrito em minúsculas ou
maiúsculas (Por exemplo: \verb/article/, \verb/ARTICLE/
e \verb/Article/ são iguais; assim como \verb/author/, \verb/AUTHOR/
e~\verb/auTHOR/).

A seguir está uma entrada bibliográfica correspondente a um livro. Os
campos obrigatórios estão todos preenchidos, e dois campos (quais?)
ignorados foram colocados também (uma anotação pessoal).

\begin{verbatim}
@BOOK{snowden,
    AUTHOR = {Edward Snowden},
     TITLE = {Vazamentos},
 PUBLISHER = {Editora Encanamentos},
      YEAR = {2013},
      NOTE = {Livro cifrado, evite divulgar a chave privada.},
   ABSTRACT = {Privacidade é um direito, que precisamos defender.
              Não dá mais para acreditar na boa-fé dos carteiros
              que carregam nossas mensagens na internet. Este é
              um livro sobre o que você e eu podemos fazer a 
              respeito.}
}

\end{verbatim}

\section{Tipos de entradas}

\begin{description}
\item[Article] 
Artigo de um periódico (\emph{journal, magazine}).

  \obrigatorios author, title, journal, year.

  \opcionais volume, number, pages, month, note, key.

\item[Book]
Livro com editora explícita, ou cuja impressão foi patrocinada por uma
organização (instituto, universidade).

  \obrigatorios author or editor, title, publisher, year.

  \opcionais volume, series, address, edition, month, note, key.

\item[Booklet]
Trabalho que foi impresso e encadernado, mas que não passou por uma
editora ou que não possui uma instiuição patrocinadora.

  \obrigatorios title.

  \opcionais author, howpublished, address, month, year, note, key.
%----------------------------------------------------------------------

\item[Conference]
Artigo nos anais ou atas de uma conferência. Este tipo de entrada é
idêntico a ``inproceedings'', e está aqui por uma questão de
compatibilidade com outros sitemas de formatação de texto.

  \obrigatorios author, title, booktitle, year.

  \opcionais editor, pages, organization, publisher, address, month, note, key.

%----------------------------------------------------------------------

\item[Inbook]
Parte de um livro; pode ser um capítulo ou um intervalo de páginas.

  \obrigatorios author or editor, title, chapter and/or pages, publisher, year.

  \opcionais volume, series, address, edition, month, note, key.

%----------------------------------------------------------------------

\item[Incollection]
Uma parte de um livro que tem um título próprio.


  \obrigatorios author, title, booktitle, year.

  \opcionais editor, pages, organization, publisher, address, month, note, key.

%----------------------------------------------------------------------

\item[Inproceedings]
Artigo nas atas ou anais de uma conferência.

  \obrigatorios author, title, booktitle, year.

  \opcionais editor, pages, organization, publisher, address, month, note, key.

%----------------------------------------------------------------------

\item[Manual]
Documentação técnica.

  \obrigatorios title.

  \opcionais author, organization, address, edition, month, year, note, key.

%----------------------------------------------------------------------

\item[Mastersthesis]
Dissertação ou tese de mestrado.

  \obrigatorios author, title, school, year.

  \opcionais address, month, note, key.

%----------------------------------------------------------------------

\item[Misc]
Use este tipo quando nada mais parecer apropriado.

  \obrigatorios 

  \opcionais author, title, howpublished, month, year, note, key.

%----------------------------------------------------------------------

\item[Phdthesis]
Tese de doutorado.

  \obrigatorios author, title, school, year.

  \opcionais address, month, note, key.

%----------------------------------------------------------------------

\item[Proceedings]
Atas ou anais (proceedings) de uma conferência.

  \obrigatorios title, year.

  \opcionais editor, publisher, organization, address, month, note, key.

%----------------------------------------------------------------------

\item[Techreport]
Relatório publicado por uma escola ou outra instituição, são geralmnte
numerados e fazem parte de uma série.

  \obrigatorios author, title, institution, year.

  \opcionais type, number, address, month, note, key.

%----------------------------------------------------------------------

\item[Unpublished]
Documento com autor e título, mas que não foi publicado formalmente.

  \obrigatorios author, title, note.

  \opcionais month, year, key.

%----------------------------------------------------------------------

\end{description}

\end{document}

% BibTeX ignores the case of letters in the entry type. 

%% R means the field is required.
%% O means it is optional.
%% A means the fields are optional, but both can't be given in the same entry.
%% E means either one or the other field must be given.
%% B means one or the other or both must be given.

%%  \begin{tabular}{cccccccccccccccccccccc}
%%  author & title & editor & booktitle & journal & volume & number & series & type & institution & school & chapter & pages & publisher & howpublished & address & edition & month & year & organization & note & key
%%  article & R & R & . & . & R & O & O & . & . & . & . & . & O & . & . & . & . & O & R & . & O & O
%%  book & E & R & E & . & . & A & A & O & . & . & . & . & . & R & . & O & O & O & R & . & O & O
%%  booklet & O & R & . & . & . & . & . & . & . & . & . & . & . & . & O & O & . & O & O & . & O & O
%%  conference & R & R & O & R & . & A & A & O & . & . & . & . & O & O & . & O & . & O & R & O & O & O
%%  inbook & E & R & E & . & . & A & A & O & O & . & . & B & B & R & . & O & O & O & R & . & O & O
%%  incollection & R & R & O & R & . & A & A & O & O & . & . & O & O & O & . & O & O & O & R & . & O & O
%%  inproceedings & R & R & O & R & . & A & A & O & . & . & . & . & O & O & . & O & . & O & R & O & O & O
%%  manual & O & R & . & . & . & . & . & . & . & . & . & . & . & . & . & O & O & O & O & O & O & O
%%  mastersthesis & R & R & . & . & . & . & . & . & O & . & R & . & . & . & . & O & . & O & R & . & O & O
%%  misc & O & O & . & . & . & . & . & . & . & . & . & . & . & . & O & . & . & O & O & . & O & O
%%  phdthesis & R & R & . & . & . & . & . & . & O & . & R & . & . & . & . & O & . & O & R & . & O & O
%%  proceedings & . & R & O & . & . & A & A & . & . & . & . & . & . & O & . & O & . & O & R & O & O & O
%%  techreport & R & R & . & . & . & . & O & . & O & R & . & . & . & . & . & O & . & O & R & . & O & O
%%  unpublished & R & R & . & . & . & . & . & . & . & . & . & . & . & . & . & . & . & O & O & . & R & O
%%  \end{tabular}
%----------------------------------------------------------------------
