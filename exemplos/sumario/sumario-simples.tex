% Based on Gonzalo Medina's answer at
% http://tex.stackexchange.com/questions/55607/page-number-before-chapter-title-in-the-table-of-contents-with-memoir

\documentclass{memoir}

% ESCREVER EM PORTUGUÊS %%%%%%%%%%%%%%%%%%%%%%%%%%%%%%%%%%%%%%%%%%%%%%
%%%%%%%%%%%%%%%%%%%%%%%%%%%%%%%%%%%%%%%%%%%%%%%%%%%%%%%%%%%%%%%%%%%%%%
\usepackage[utf8]{inputenc}
\usepackage[T1]{fontenc}
\usepackage[brazil]{babel}

% PARA ESCREVER ENDEREÇOS NA INTERNET %%%%%%%%%%%%%%%%%%%%%%%%%%%%%%%%
% o parâmetro `hyphens` faz com que links \url{www.wikipedia.org}
% possam ser hifenados.
\usepackage[hyphens]{url}

% FORMATAÇÃO DO SUMÁRIO %%%%%%%%%%%%%%%%%%%%%%%%%%%%%%%%%%%%%%%%%%%%%%
%%%%%%%%%%%%%%%%%%%%%%%%%%%%%%%%%%%%%%%%%%%%%%%%%%%%%%%%%%%%%%%%%%%%%%

% Redefinir estilo do título do sumário
\renewcommand\printtoctitle[1]{\HUGE#1}

% redefinitions for chapter entries
\renewcommand\chapternumberline[1]{}

% nome do capítulo escrito grande
\renewcommand\cftchapterfont{\LARGE} 
% número da página escrito grande e em itálico
\renewcommand\cftchapterpagefont{\LARGE \itshape} 


% Formatação de cada linha do sumário correspondendo a um capítulo
% %%%%%%%%%%%%%%%%%%%%%%%%%%%%%%%%%%%%%%%%%%%%%%%%%%%%%%%%%%%%%%%%%%%%
% Estamos definindo aqui o comando \l@mychap
% 
\makeatletter
\newcommand\l@mychap[3]{%
  \bigskip% pulo vertical
  \par\noindent% novo parágrafo, sem indentação
  \parbox{\dimexpr\textwidth-15pt\relax}{%
    \cftchapterfont#1, {\cftchapterpagefont#2}%
  }\par%
}

% Este é o comando que o memoir usa, escrevemos ele
% de modo que use o comando \l@mychap acima.
\renewcommand*\l@chapter[2]{%
  \l@mychap{#1}{#2}{\chaptername}%
}
\makeatother
% %%%%%%%%%%%%%%%%%%%%%%%%%%%%%%%%%%%%%%%%%%%%%%%%%%%%%%%%%%%%%%%%%%%%


% DOCUMENTO %%%%%%%%%%%%%%%%%%%%%%%%%%%%%%%%%%%%%%%%%%%%%%%%%%%%%%%%%%
%%%%%%%%%%%%%%%%%%%%%%%%%%%%%%%%%%%%%%%%%%%%%%%%%%%%%%%%%%%%%%%%%%%%%%
\begin{document}

\noindent{\LARGE Formatação do sumário usando memoir}\hfill 8 de março de 2016

\medskip

\noindent Tássio Naia dos Santos

\medskip

% Anote aqui se fizer modificações, usando o seguinte formato
%
%\noindent dd/mm/aaaa: mudança (queam)
%
% Por exemplo:
%
%\noindent 30/2/2222: correções gramaticais (Lucas Silva e Silva)


\bigskip

\begin{abstract}
Aqui está um sumário modificado.
A única alteração é o~modo como o~sumário formata linhas 
que correspondem a~capítulos (\cs{chapter}). 
Seções (\cs{section}) continuam formatadas 
do jeito padrão.
Lembre-se de que para atualizar o sumário é preciso compilar
o~documento duas vezes.
\end{abstract}

\tableofcontents*

\chapter{Pela toca do coelho}
\chapter{O lago das lágrimas}
\chapter{A corrida e a longa história}
% mudar o número da página para o exemplo
\setcounter{page}{14}

\chapter{Um título que é longo, mas tão longo, que algumas pessoas vão achar longo demais}
\chapter{Pílulas do pé-de-vento}
\section{Tradição, uma seção}
\section{Bis!}
\section{Três para dar sorte}

\chapter{F\dotfill im}

\chapter*{Agradecimentos \& crédito}

Este exemplo é baseado na resposta que Gonzalo Medina escreveu em \url{http://tex.stackexchange.com/questions/55607/page-number-before-chapter-title-in-the-table-of-contents-with-memoir}.

\end{document}
