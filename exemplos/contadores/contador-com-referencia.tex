%%%%%%%%%%%%%%%%%%%%%%%%%%%%%%%%%%%%%%%%%%%%%%%%%%
% Autor: Tássio Naia
% Criado: 16-03-16
% Modificado: 16-03-16
%%%%%%%%%%%%%%%%%%%%%%%%%%%%%%%%%%%%%%%%%%%%%%%%%%
\documentclass{memoir}


%%%%%%%%%%%%%%%%%%%%%%%%%%%%%%%%%%%%%%%%%%%%%%%%%%
% Escrevendo em português
\usepackage[utf8]{inputenc}
\usepackage[T1]{fontenc}
\usepackage[brazil]{babel}

%%%%%%%%%%%%%%%%%%%%%%%%%%%%%%%%%%%%%%%%%%%%%%%%%%
% Cria contador de receitas
\newcounter{receita}

%%%%%%%%%%%%%%%%%%%%%%%%%%%%%%%%%%%%%%%%%%%%%%%%%%
% Cria ambiente usando o contador
%
% Explicação:
% O ambiente começa avançando o contador receita.
% Usamos \refstepcounter em vez de \stepcounter
% porque vamos escolher rótulos (\label) para
% as receitas.
%
% Depois iniciamos um novo parágrafo (\par),
% deixando um espaço vertical médio (\medskip),
% e escrevemos "Receita #X" em negrito;
% para terminar, depois de cada receita
% iniciamos outro parágrafo.
\newenvironment{receita}{%
  \refstepcounter{receita}%
  \par\medskip\noindent\textbf{Receita \thereceita.}}{%
  \par}

%%%%%%%%%%%%%%%%%%%%%%%%%%%%%%%%%%%%%%%%%%%%%%%%%%
\begin{document}

\noindent{\LARGE Contadores e referências}\hfill 16 de março de 2016

\medskip

\noindent Tássio Naia Tandonnet

\medskip

\noindent 16/03/2016: primeira versão.

\bigskip

\begin{abstract}
  Este é um exemplo de como usar contadores no \LaTeX\
  de modo que seja possível referenciar os números
  usando os comandos~\cs{ref} (e~\cs{label}).
\end{abstract}

\TeX nicalidades: o comando \cs{newcounter} cria um contador novo.
Use \cs{refstepcounter} para aumentar o~valor do~contador.
O ambiente ``receita'' foi criado assim:
\begin{verbatim}
\newenvironment{receita}{%
  \refstepcounter{receita}%
  \par\medskip\noindent\textbf{Receita \thereceita.}}{%
  \par}
\end{verbatim}

Aqui estão alguns comentários sobre
receitas culinárias. As receitas são numeradas,
e podem ser referenciadas: por exemplo---recomendo
bastante o~suco (receita~\ref{suco}) e também
o~amendoim (receita~\ref{amendoim})!

\begin{receita}
  Bolo de banana e cenoura. Uma delícia!
  Ainda me lembro do dia em que minha tia
  preparou para a gente esse bolo, perto
  do natal. Não sobrou nem farelo.
\end{receita}

\begin{receita}\label{amendoim}
  Amendoim torrado. Não abuse do sal.
  Esse é um favorito dos meus tios.
  No inverno, minha avó prepara uma
  bacia cheia, e tardes inteiras são
  dedicadas à degustação e um bom
  papo.
\end{receita}

\begin{receita}\label{suco}
  Vitamina de beterraba, cenoura e laranja.
  A favorita da garotada! Lembro bem da
  vitamina no meio da manhã, no verão.
  A gente corria tanto, e suava tanto!
  Quando vinha o chamado pro suco a gente
  corria ainda mais. Que refrescante
  essa lembrança.
\end{receita}


\end{document}
