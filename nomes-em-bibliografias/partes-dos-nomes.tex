\documentclass[article,openany]{memoir}

% Codificação da entrada e da saída ==================================
\usepackage[utf8]{inputenc}
\usepackage[T1]{fontenc}

% Idioma =============================================================
\usepackage[brazil]{babel}

% Hiperlinks =========================================================
\usepackage{url}

% Tipografia =========================================================
% (Podem ser removidos sem problema.)
\usepackage{lmodern}
\usepackage{microtype}

% Autor e título =====================================================
\title{Bib\TeX: nomes ácèñ\d{t}ü\v{a}çã\b{o}}
\author{Tássio Naia dos Santos}

% Abreviações ========================================================
\newcommand{\obrigatorios}{%
   \par Obrigatório\dotfill\ignorespaces}

\newcommand{\opcionais}{
   \par Opcional\dotfill\ignorespaces}

\newcommand*{\extensao}[1]{\texttt{#1}}

\newcommand*{\astt}[1]{% <--- aspas simples
  `\thinspace\texttt{#1}\thinspace'} 

\newcommand*{\as}[1]{% <--- aspas simples
  `\thinspace#1\thinspace'}

\newenvironment{exemplo}{\begin{center}}{\end{center}}
% Deixando a coluna de texto um pouco mais gorda =====================
\settypeblocksize{23cm}{13cm}{*}
\setulmargins{*}{*}{1}
\setlrmargins{*}{*}{1}
\checkandfixthelayout

\begin{document} % ===================================================


%\maketitle

\noindent{\LARGE Bib\TeX: nomes e ácèñ\d{t}ü\v{a}çã\b{o}}\hfill 27 de julho de 2014

\medskip

\noindent Tássio Naia dos Santos

%\medskip

%\noindent 25/07/2014: correções gramaticais (Shayenne Luz Moura).

\bigskip

\begin{abstract}
Apresentamos as partes de um nome --- como percebidas e tratadas pelo
Bib\TeX. Também descrevemos um modo para obter a ordem alfabética
adequada para entradas que contém caracteres diferentes das 26 letra
do alfabeto inglês.
\end{abstract}


\chapter*{Agradecimentos \& crédito}

Agradecemos a Norman Walsh\footnote{\url{http://nwalsh.com/tex/}}, de
onde obtivemos parte deste material. A tradução e os erros são de
responsabilidade de~Tássio Naia dos~Santos.



\chapter{Acentuação}

Existem diversas maneiras de colocar acentos e outros sinais
diacríticos sobre letras no \LaTeX. Com a popularização do Unicode,
mais e mais pessoas são capazes de escrever textos no computador
usando um teclado nativo para seu idioma. Para a língua portuguesa,
isso significa que, usando a combinação adequada de teclas, somos
capazes de inserir caracteres ``acentuados'', como \astt{í}, \astt{c},
\astt{ã}, \astt{ê}, etc.

Quando o \TeX\ foi criado, não havia sequer internet. Uma herança
dessa época é que existem modos de escrever palavras acentuadas usando
comandos, o que permite obter uma grande gama de letras usando apenas
um conjunto de caracteres básico e largamente presente nos teclados
ocidentais, como as letras de `a' a~`z', e sinais como aspas
\astt{\char`\`} e~\astt{\char`\'}, circunflexo~\astt{\char`\^},
til~\astt{\char`\~} acento grave~\astt{\char`\`} e aspas duplas~\astt{\char`\"}.

Uma vantagem dos uso de comandos para inserir sinais diacríticos é
que, ao separar o acento da ``letra'' que ele modifica\footnote{Aqui,
  quando dizemos ``letra'', estamos falando das letras a, b, c,\ldots,
  z.}, fica mais simples (para o computador) fazer a ordenação
alfabética de uma lista de nomes. O computador ignora os acentos, e
considera apenas a sequência de ``letras'' do nome.

Por exemplo, para escrever a ligadura tipográfica em ``\oe vre'',
usa-se o comando~\cs{oe}. Note que quando o comando aparece ao fim de
uma palavra, será necessário, por por vezes, usar a sequência de
comando barra-espaço \astt{\textbackslash\textvisiblespace}. Compare
\begin{exemplo}
\begin{tabular}{lll}
\verb/Arch\ae é uma\ldots/ & resulta em & Arch\ae é uma\ldots \\
\verb/Arch\ae\ é uma\ldots/ & resulta em & Arch\ae\ é uma\ldots
\end{tabular}
\end{exemplo}

A tabela abaixo exemplifica alguns dos vários sinais que podemos
escrever com o~\LaTeX.
\begin{exemplo}
\begin{tabular}{rcl}
 \verb/\`{o}/ & \`{o} & acento grave                                   \\
 \verb/\'{o}/ & \'{o} & acento agudo                                   \\
 \verb/\^{o}/ & \^{o} & acento circunflexo                             \\
 \verb/\"{o}/ & \"{o} & trema                                          \\
 \verb/\H{o}/ & \H{o} & trema Húngaro longo (duplo agudo)              \\
 \verb/\~{o}/ & \~{o} & acento til                                     \\
 \verb/\c{c}/ & \c{c} & cedilha                                        \\
 \verb/\={o}/ & \={o} & acento mácron (barra sobre a letra)            \\
 \verb/\b{o}/ & \b{o} & barra sob a letra                              \\
 \verb/\.{o}/ & \.{o} & ponto sobre a letra                            \\
 \verb/\d{u}/ & \d{u} & ponto sob a letra                              \\
 \verb/\r{a}/ & \r{a} & anel (anel sobre a letra)                      \\
 \verb/\u{o}/ & \u{o} & bráquia (`u' sobre a letra)                    \\
 \verb/\v{s}/ & \v{s} & caron ou há\v cek (`v' sobre a letra)          \\
 \verb/\t{oo}/ 
              & \t{oo}     & `u' invertido sobre duas letras           \\
\verb/\oe/ e \verb/\OE/ 
              & \oe\ e \OE & ligadura tipográfia francesa              \\
\verb/\ae/ e \verb/\AE/
              & \ae\ e \AE & ligadura tipográfica latina e escandinava \\
\verb/\aa/ e \verb/\AA/ 
              & \aa\ e \AA & `a' com anel escandinavo                  \\
\verb/\o/ e \verb/\o/ 
              & \o\ e \O   & `o' barrado escandinavo                   \\
\verb/\l/ e \verb/\L/ 
              & \l\ e \L   & L-suprimido polonês                       \\
\verb/\ss/ & \ss           & ``es-zet'' alemão
\end{tabular}
\end{exemplo}
% O trecho abaixo vem de nwalsh.com/tex/texhelp/bibtx-23.html
% Fiz pequenas alterações, e adicionai exemplos em português.

Note que as letras \as{i} e~\as{j} precisam de tratamento especial, já
que é preciso \emph{remover o pingo} antes de acrescentar o acento:
`\verb/\^\i/' produz~\as{\^\i}, e~`\verb/\`\j/' produz \as{\`\j'}.

\chapter{Nomes}

O texto dos campos autor e editor em um arquivo \extensao{.bib}
representa uma lista de nomes. O estilo da bibliografia determina em
que formato o nome deve ser impresso: se é o primeiro ou o último nome
a vir primeiro, se o primeiro aparece por extenso ou abreviado, etc. O
arquivo de bibliografia simplesmente diz a o Bib\TeX\ qual é o nome.

Você deve escrever o nome completo do autor e deixar o estilo da
bibliografia decidir o que deve ser abreviado. (Mas o nome completo de
um autor pode, a princípio, ser ``Donald~E. Knuth'' ou mesmo
``J.~P.~Morgan''; você deveria escrevê-lo do modo que o autor gostaria
que ele aparecesse, se isso é conhecido.)  to appear, if that's
known.)

A maior parte dos nomes pode ser escrita do modo óbvio, com ou sem uma vírgula, como nos exemplos a seguir.
\begin{exemplo}
\begin{tabular}{p{5cm}l}
John Paul Jones  &  Ludwig von Beethoven \\
Jones, John Paul &  von Beethoven, Ludwig
\end{tabular}
\end{exemplo}


Algumas pessoas possuem sobrenomes múltiplos --- por exemplo, o
sobrenome de ``Per Brinch Hansen'' é ``Brinch~Hansen''. Este nome deve
ser escrito (no arquivo \extensao{.bib} com uma vírgula:

\begin{exemplo}
Brinch Hansen, Per
\end{exemplo}

Outro exemplo, para ``Adriano Moreira Passos'':

\begin{exemplo}
Moreira Passos, Adriano
\end{exemplo}

Para entender o porquê, é preciso compreender o modo como o
Bib\TeX\ lida com nomes (no restante desse artigo, ``nome''
corresponde ao nome de uma pessoa).

Cada nome consiste de quatro partes: Primeiro, von, Último, e~Jr; cada
parte consiste de uma lista (possivelmente vazia) de pedaços.  Por
exemplo, a primeira parte de Adriano Moreira~Passos é ``Adriano''; a
última possui duas partes ``Moreira'' e~``Passos''; e as partes
``von'' e ``Jr'' estão vazias. Se você tivesse escrito

\begin{exemplo}
Adriano Moreira Passos
\end{exemplo}

sem a vírgula, o Bib\TeX teria assumido que ``Moreira'' era uma parte
do primeiro nome, assim como ``Paulo'' é uma parte do primeiro nome de
``Jõao~Paulo Prado''.

Aqui está um outro exemplo:

\begin{exemplo}
Charles Louis Xavier Joseph de la Vallée Poussin
\end{exemplo}

Este nome possui quatro partes no primeiro nome, duas no von, e duas
no sobrenome. Aqui o Bib\TeX\ sabe onde cada parte termina e outra
inicia porque as partes na parte von começam com letras minúsculas.

Se você quiser que o Bib\TeX\ considere algo como uma única parte,
basta colocar a expressão entre chaves (isto é, entre ``\{''
e~``\}''). Você deve fazer isso, por exemplo, se uma vírgula é parte
de um nome:

\begin{exemplo}
\begin{minipage}{5cm}
\{Barnes and Noble, Inc.\}

\{Barnes and\} \{Noble, Inc.\}

\{Barnes\} \{and\} \{Noble,\} \{Inc.\}
\end{minipage}
\end{exemplo}

As chaves que envolvem a vírgula impedem que ``Inc.'' seja
interpretado como uma parte do primeiro nome; este nome possui apenas
uma parte no sobrenome, que

 this name has only a Last part, with either one, two, or four tokens
 (there must be a space separating the tokens in the second and third
 forms). Probably the second form is slightly more meaningful, but
 don't lose sleep over this since only rarely will an institution be
 an author or editor.

Os dois nomes
\begin{exemplo}
von Beethoven, Ludwig\quad\ e\quad \{von Beethoven\}, Ludwig
\end{exemplo}
são vistos diferentemente pelo Bib\TeX. No primeiro, ``Beethoven'' é o
sobrenome e ``von'' é a parte von; no segundo, que neste caso está
incorreto, o sobrenome apenas uma palavra e não existe a parte von. O
estilo de bibliografia provavelmente imprimirá ambos da mesma forma,
mas isso afetará ordem afabética.

Além de ``de la (Valée)'' e ``von~(Beethoven)'', outros exemplos de
nomes com parte von são ``de~Almeida'' e ``dos~Santos''.


``Juniors'' é um caso em particular. A maior parte das pessoas que
possuem ``Jr.'' em seu nome têm por hábito precedê-lo por uma
vírgula. Tais nomes devem ser escritos como segue:

\begin{exemplo}
Ford, Jr., Henry
\end{exemplo}

Algumas poucas pessoas, contudo, não usam a vírgula. Neste caso você
pode considerar o ``Jr.'' como parte do sobrenome:

\begin{exemplo}
\{Steele Jr.\}, Guy L.\quad\ e\quad ``Guy L. \{Steele Jr.\}
\end{exemplo}
O Bib\TeX\ não considerará que esse nome possui a parte Jr.

\smallskip

Resumindo, você pode escrever um nome em uma das três formas a seguir:
\begin{itemize}
\item Primeiro von Sobrenome
\item von  Sobrenome, Primeiro
\item von Sobrenome, Jr, Primeiro
\end{itemize}

Você pode quase sempre usar a primeira forma; mas é melhor evitá-la se
existe uma parte Jr, ou se a última parte possui mais do que uma
palavra e não existe parte von.

Se existem múltiplos autores ou editores, seus nomes devem ser
separados pela palavra ``and''. O ``and'' deve estar entre espaços, e
\emph{não} deve estar entre chaves.

\begin{exemplo}
AUTHOR = ``Ralph Alpher and Bethe, Hans and George Gamow''
\end{exemplo}

Como o Bib\TeX\ interpreta vírgulas como um separador das partes de um
nome, e ``and'' como separador de nomes, o exemplo acima indica três
co-autores. Ralph Alpher, Hans Bethe, e George Gamow. Se a palavra
``and'' aparece como parte de um nome, ela deve estar entre caves,
como no exemplo ``Barnes and Noble,~Inc.'' acima. Se você tem nomes
demais para listar em um campo, ccovê pode terminar a lista com ``and
others''; a maioria dos estilos de bibliografia trata substitui isso
por~``et al.''

As regras do Bib\TeX\ são, na verdade, um pouco mais complicadas do
que indicado aqui, mas esta descrição basta para a maior parte dos
nomes.

\chapter{Exercícios}

Tente descobrir como escrever os seguintes nomes para que eles sejam
ordenados de modo correto alfabeticamente.
\begin{enumerate}
\item Pál Erd\H os
\item João Paulo Condé de Oliveira Prado
\item Jean de La Fontaine
\item Pedro de Alcântara Francisco António João Carlos Xavier de Paula Miguel Rafael Joaquim José Gonzaga Pascoal Cipriano Serafim de Bragança e Bourbon
\item Alexandra David-Néel
\item Elisabeth Marie Monique Suzanne Sophie Delphine Voldrue Tandonnet
\item L\ae s\o\ von Neumann Jr.
\item Caius Iulius C\ae sar IV
\end{enumerate}

\end{document}
