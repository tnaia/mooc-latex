\documentclass[12pt,a4paper,article]{memoir}

\usepackage[utf8]{inputenc}
\usepackage[T1]{fontenc}
\usepackage[brazil]{babel}

\usepackage{lmodern,microtype}

\begin{document}
\title{Fábulas}
\author{Esopo}
\maketitle

\chapter{O Lobo e o Cordeiro}

Estava bebendo um Lobo encarniçado à beira de um riacho, e pela parte
debaixo chegou um Cordeiro, também a beber. Olhou-o o Lobo de má cara,
e disse arreganhando os dentes: Porque tiveste tanta ousadia de me
turvar a água onde estou bebendo? Respondeu o Cordeiro com humildade:
a água corre para mim, e portanto não posso turvar a vossa. Torna o
lobo, mais colérico, a dizer: Não me venha com desculpas! Seis meses
atrás me fazia o mesmo o teu Pai. Respondeu o Cordeiro: Neste tempo
Senhor, ainda eu não era nascido, não tenho culpa. Sim tens, replicou
o Lobo, que todo o pasto de meu campo estragaste. Mal pode ser isso,
disse o Cordeiro, porque ainda não tenho dentes. O Lobo, sem mais
razões, saltou sobre ele, e logo o degolou e comeu.


\section{Moralidade}

Mostra esta Fábula que nenhuma justiça, nem razões valem ao inocente,
para o livrarem das mãos do inimigo poderoso e desalmado. Poucas
cidades ou vilas há onde não haja estes Lobos, que sem causa nem
razão, matam ao pobre e chupam-lhe o sangue.

\chapter{O Leão e o Rato}

Estando o Leão dormindo, andavam uns ratos brincando ao redor dele, e
saltando-lhe por cima o acordaram. Tomou ele um entre as patas estava
para o matar; pelos rogos deste, o soltou. Sucedeu daí a pouco cair o
Leão em uma rede, onde ficou liado, sem poder valer-se de suas
forças. Sabendo-o o Rato, tal diligêcia pôs, que roeu brevemente os
laços e cordéis, e soltou o Leão que se foi livre em pago da boa obra
que lhe fez.

\section{Moralidade}

Duas coisas temos aqui que notar: primeiramente o agradecimento que se
deve a qualquer boa obra, e em especial a quem perdoa algum agravo,
podendo vingar-se como este Leão podia. Secundariamente, quanto devem
os poderosos estimar a amizade de qualquer homem, por fraco que seja;
porque qualquer pode fazer mal, e se não podem fazer mal, todos podem
fazer bem.
\end{document}
