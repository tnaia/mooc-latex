\documentclass[a5paper,11pt,article,oneside]{memoir}
\usepackage[utf8]{inputenc}
\usepackage[T1]{fontenc}
\usepackage[brazil]{babel}
\usepackage{lmodern,microtype}
\usepackage{tikz}

\usepackage[colorlinks,%
  linkcolor=black,%green!40!blue,%
  citecolor=blue,%
  urlcolor=red,%
  pdfauthor={T\'assio Naia}%
  ]{hyperref}

% Redefinir estilo do título do sumário
\renewcommand\printtoctitle[1]{}%\Large\bfseries#1}

% redefinitions for chapter entries
\renewcommand\chapternumberline[1]{}

% nome do capítulo escrito grande
\renewcommand\cftchapterfont{\large\scshape} 
% número da página escrito grande e em itálico
\renewcommand\cftchapterpagefont{\large \itshape\color{red!80!black}} 

% Formatação de cada linha do sumário correspondendo a um capítulo
% %%%%%%%%%%%%%%%%%%%%%%%%%%%%%%%%%%%%%%%%%%%%%%%%%%%%%%%%%%%%%%%%%%%%
% Estamos definindo aqui o comando \l@mychap
% 
\makeatletter
\newcommand\l@mychap[3]{%
  \smallskip% pulo vertical
  \par\noindent% novo parágrafo, sem indentação
  \parbox{\dimexpr\textwidth-15pt\relax}{%
    \cftchapterfont#1\quad {\cftchapterpagefont#2}%
  }\par%
}

% Este é o comando que o memoir usa, escrevemos ele
% de modo que use o comando \l@mychap acima.
\renewcommand*\l@chapter[2]{%
  \l@mychap{#1}{#2}{\chaptername}%
}
\makeatother
% %%%%%%%%%%%%%%%%%%%%%%%%%%%%%%%%%%%%%%%%%%%%%%%%%%%%%%%%%%%%%%%%%%%%

\let\velhoEmph\emph
\renewcommand{\emph}[1]{\textcolor{blue!80!black}{\velhoEmph{#1}}}

% Margens %%%%%%%%%%%%%%%%%%%%%%%%%%%%%%%%%%%%%%%%%%%%%%%%%%%%%%%%%%%%%%
\settypeblocksize{16cm}{10cm}{*}% vertical, horizontal

\setlrmargins{*}{*}{1}
\setulmargins{*}{*}{1}
\checkandfixthelayout


\begin{document}

\hfill{\HUGE Um curso de \color{red!80!black}\LaTeX}
\bigskip

\tableofcontents*

\hypersetup{linkcolor=red!90!black}

\bigskip

\chapter{Introdução}
Aqui temos um exemplo de documento preparado usando \LaTeX.
Ele é um \emph{pot-pourrit} das várias coisas que vamos abordar
durante o curso. 

Os documentos são produzidos em duas etapas, partindo do texto-fonte, 
onde estão o~conteúdo do~texto (frases e parágrafos) e~também 
as~instruções para sua~formatação.
Essas instruções explicam ao ao~computador como formatar o~texto---são
\emph{marcações} que indicando o~papel de cada trecho do texto. A grosso modo,
isso nada mais é do que um jeito de dizer qual é a parte do texto que 
corresponde a um título, uma tabela, uma imagem, etc.


\chapter{Vantagens}

\LaTeX\ é um sistema de tipografia sofisticado, que permite
produzir documentos de grande qualidade tipográfica.
Ele automatiza~a
\begin{enumerate}
\item divisão silábica;
\item numeração de páginas, capítulos, tabelas, etc.;
\item construção do índice, das listas de figuras e de tabelas;
\item ordenação e formatação da bibliografia; e a
\item formatação de equações.
\end{enumerate}
% estável
O \LaTeX\ é um sistema \emph{robusto}. Você pode usá-lo com
confiança para produzir documentos longos 
(com centenas ou milhares de páginas\ldots\ o~\LaTeX\ não vai travar,
 nem perder o seu texto!). 
O programa também é \emph{estável}: seus documentos poderão ser lidos 
daqui a~5, 
10~ou 50~anos\footnote{Isso acontece por duas razões. O~conteúdo (texto-fonte)
é armazenado no computador como texto puro (plain), que sempre será legível.
A~segunda razão é que o~sistema de base do~\LaTeX, 
chamado~\TeX, foi congelado. Ele nunca mais vai mudar! 
Isso garante a~estabilidade. Por incrível que pareça, isso não significa que ele
vai ficar para trás, já que é possível \emph{estender} 
sua~funcionalidade.}.
Existe ainda uma terceira vantagem, um pouco mais difícil de explicar.
É bastante simples criar estilos de documento com o~\LaTeX.
Isso nos permite reutilizar a~formatação dos documentos!%
\footnote{A~vantagem é clara se pensamos em um conjunto de relatórios, 
boletins edições de uma~revista, livros de uma mesma coleção, teses de uma 
universidade, etc.}
Essas são grandes vantagens, especialmente do~ponto de~vista 
do~arquivamento de~textos e da produção
de vários documentos com formatação similar. 
Não é surpresa que este seja o padrão para vários setores da~publicação
acadêmica!

\chapter{Software livre}

Talvez uma das maiores vantagens do~\LaTeX\ é que ele é
\emph{software livre}. Isso significa que ele não aprisiona seus usuários:
qualquer pessoa tem a~permissão de~``abrir o capô'' do programa 
para ver como ele funciona, de~adaptá-lo às suas necessidades (o~que 
inclui corrigir falhas que encontrar) e redistribuir o programa (com ou sem
as modificações) para que outras pessoas beneficiem dele. Essas são
as 4~liberdades de um software livre:


\begin{description}
\item[\small Liberdade 0.] Executar o programa com qualquer propósito;
\item[\small Liberdade 1.] Estudar o programa;
\item[\small Liberdade 2.] Redistribuir cópias do programa de modo que você possa ajudar ao seu próximo; e
\item[\small Liberdade 3.] Modificar o programa e distribuir estas modificações, de modo que toda a comunidade se beneficie.
\end{description}
A Fundação do Software Livre (FSF, veja Figura~\ref{fig:logo-fsf})
há mais de 30 anos discute e fomenta a~produção
de~software livre. Ela é uma grande representante desse movimento,
defendendo liberdade e~democracia nos meios digitais.
O PoliGNU, que está oferecendo este curso, é um grupo de estudo 
e discussão de software e cultura livres.

\begin{figure}
\centering
\includegraphics[width=3cm]{FSF_logo_menor}%
\caption{Logotipo da Fundação do Software Livre.}\label{fig:logo-fsf}
\end{figure}

\chapter{Conteúdo do curso}

Ao fim do curso teremos trabalhado com os
elementos básicos de um texto.

\begin{itemize}
\item formatação do texto;
\item formatação da página;
\item tabelas e figuras;
\item expressões matemáticas; e
\item referências bibliográficas.
\end{itemize}

%% \begin{itemize}
%% \item índce
%% \item nota de rodapé
%% \item figura
%% \item tabela
%% \item expressão matemática
%% \end{itemize}

%% \begin{itemize}
%% \item relatório (referências bibliográficas)
%% \item formulário (estilo carta com assinatura)
%% \item tufte-style (notas de aula)
%% \item slides
%% \item diagrama com tikz
%% \item chemfig
%% \item tese
%% \end{itemize}

\end{document}
