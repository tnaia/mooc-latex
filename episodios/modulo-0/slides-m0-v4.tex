%todo: foto knuth e lamport
\documentclass{beamer}
\usepackage[utf8]{inputenc}
\usepackage[T1]{fontenc}
\usepackage[brazil]{babel}
\usepackage{tikz}
\usepackage{url}
\newcommand{\figura}[2]{%
  \tikz[baseline={(0,0)}]
  \node at (0,0) {\includegraphics[width=#1]{#2}};}

\newcommand{\site}[2]{%
  \parbox{#1}{\centering\url{#2}}}

\setbeamertemplate{navigation symbols}{}% remove navigation symbols

\begin{document}

\begin{frame}{Donald Knuth e \TeX}

\begin{center}
  \figura{2cm}{1998_donald_knuth}\quad
  \Large Tipografia para a era digital \pause
  \hfill{\small\textcolor{red!80!black}{1978!!}}
\end{center}

\pause

\bigskip

\begin{itemize}
\item linguagem para formatação de textos (\TeX),
\item programa (software) sofisticado
\item libertou os programas -- \TeX\ é \alert{Software Livre}
\end{itemize}
\end{frame}

\begin{frame}{Leslie Lamport e o \LaTeX}

Desvantagem do \TeX: tipografia é difícil!

\pause\medskip

\figura{3cm}{leslie-lamport_vignette}\quad
Lamport fez o \LaTeX\ (usando o \TeX)\hfill{\small\textcolor{red!80!black}{1985}}

\pause\bigskip

\begin{itemize}
\item mais fácil usar que o \TeX
\item vários estilos predefinidos 
\item também é \alert{Software livre}
\end{itemize}
\end{frame}

\begin{frame}{Software Livre}

\begin{center}
Programas que respeitam quem usa.


\pause\bigskip

\end{center}


\alert{4 Liberdades do software livre}:

\pause\medskip

\begin{description}
\item[\small Liberdade 0.] \alert{Executar} o programa com qualquer propósito;\pause
\item[\small Liberdade 1.] \alert{Estudar} o programa;\pause
\item[\small Liberdade 2.] \alert{Redistribuir} cópias do programa de modo que você possa ajudar ao seu próximo; e\pause
\item[\small Liberdade 3.] \alert{Melhorar} o programa e distribuir estas modificações, de modo que toda a comunidade se beneficie.
\end{description}

\pause
\begin{center}
Livre nem sempre é gratuito.
\end{center}
\end{frame}

\begin{frame}{Software Livre por toda parte}

  \begin{center}
    \Large Estamos rodeados de software livre
  \end{center}

\pause\bigskip

\begin{itemize}[<+->]
\item GNU/Linux --- sistemas operacionais livres (ex: ubuntu, triskel, archlinux)
\item Firefox --- navegador livre
\item Blender --- modelagem e animação 3D
\item MediaWiki --- software usado na Wikipédia
\end{itemize}
\end{frame}

\begin{frame}{FSF, CCSL e PoliGNU}

\begin{center}
\Large Software \alert{livre} por um mundo \alert{melhor}.
\end{center}

\pause

\hfill\hfill\figura{2.6cm}{FSF_logo_menor.jpg}
\pause\hfill\figura{3.2cm}{ime-usp-ccsl.png}
\pause\hfill\figura{3.2cm}{polignu.png}
\hfill\hfill


\pause

\hfill\hfill\site{2.6cm}{fsf.org}
\hfill\site{3.2cm}{ccsl.ime.usp.br}
\hfill\site{3.2cm}{polignu.org}
\hfill\hfill


\end{frame}

\end{document}
