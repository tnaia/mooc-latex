\documentclass[article,openany]{memoir}

% Codificação da entrada e da saída ==================================
\usepackage[utf8]{inputenc}
\usepackage[T1]{fontenc}

% Idioma =============================================================
\usepackage[brazil]{babel}

% Hiperlinks =========================================================
\usepackage{url}

% comando \text para texto em fórmulas ===============================
\usepackage{amsmath}

% Tipografia =========================================================
% (Podem ser removidos sem problema.)
\usepackage{lmodern}
\usepackage{microtype}

% Autor e título =====================================================
\title{Matemática: o tamanho de delimitadores}
\author{Tássio Naia dos Santos}

% Abreviações ========================================================
\newcommand{\obrigatorios}{%
   \par Obrigatório\dotfill\ignorespaces}

\newcommand{\opcionais}{
   \par Opcional\dotfill\ignorespaces}

\newcommand*{\extensao}[1]{\texttt{#1}}

\newcommand*{\pacote}[1]{\textsf{#1}}

\newcommand*{\ambiente}[1]{\textsf{#1}}

\newcommand*{\astt}[1]{% <--- aspas simples com texto monoespaçado
  `\thinspace\texttt{#1}\thinspace'} 

\newcommand*{\as}[1]{% <--- aspas simples com texto normal
  `\thinspace#1\thinspace'}

\newenvironment{exemplo}{\begin{center}}{\end{center}}
% Deixando a coluna de texto um pouco mais gorda =====================
\settypeblocksize{23cm}{13cm}{*}
\setulmargins{*}{*}{1}
\setlrmargins{*}{*}{1}
\checkandfixthelayout

\begin{document} % ===================================================


%\maketitle

\noindent{\LARGE Matemática: o tamanho de delimitadores}\hfill 28 de julho de 2014

\medskip

\noindent Tássio Naia dos Santos

% últimas modificações ===============================================
%\medskip
%\noindent 25/07/2014: correções gramaticais (Shayenne Luz Moura).

\bigskip

\begin{abstract}
Listamos alguns dos delimitadores de expressões matemáticas
(parênteses, colchetes, piso, teto, módulo, etc.) e como controlar seu tamanho em
expressões matemáticas.
\end{abstract}

\chapter{Controle ``automático'' do tamanho de delimitadores}

Alguns símbolos mudam de tamanho de acordo com o modo matemático
(parágrafo ou destaque) usado para escrevê-los. Por exemplo,
compare~\(\sum_i f(i)\) e
\[
\sum_i f(i).
\]

Quando estamos compondo expressões mais complexas, digamos, com termos
agrupados entre parênteses, podemos aumentá-los, parad a dar mais
clareza à expressão:
\[
\left[\sum_i^{30} f(i)\right] ^2; 
\quad \text{é mais claro que}\quad 
[\sum_i f(i)]^2
\]

O resultado acima (à esquerda) é obtido usando o par de comandos \cs{right}
e~\cs{left}:
\begin{exemplo}
\verb/\left[  \sum_i^{30} f(i)  \right] ^2/.
\end{exemplo}
Ao colocar \cs{left} e \cs{right} precedendo os delimitadores que
envolvem uma expressão, eles são aumentados automaticamente. Assim,
\begin{exemplo}
\verb/\[ \left\lfloor \frac{ x^2 }{3} \right\rfloor \]/ produz 
\(\displaystyle \left\lfloor \frac{ x^2 }{3} \right\rfloor \).
\end{exemplo}

Os comandos \cs{left} e~\cs{right} aparecem sempre aos pares, e uma
expressão pode conter mais de um par deles: 
\begin{exemplo}
\verb/\[\left( \left( x^2 \right)^3 \right).\]/
produz
\(\displaystyle \left( \left( x^2 \right)^3 \right).\)
\end{exemplo}
%produz
%\[\left( \left( x^2 \right)^3 \right).\]
que é bem diferente do resultado que obtamos sem usar \cs{left}
e~\cs{right}: \(\displaystyle ( ( x^2 )^3 ).\)

Quando desejamos apenas um dos delimitadores, é possível usar um ponto
\astt{.} para omitir o outro. Por exemplo
\begin{exemplo}
\verb/\[ \left. \frac{ax^2}{b} \right|_a^b \]/
produz
\(\displaystyle
\left. \frac{ax^2}{b} \right|_a^b \).
\end{exemplo}
onde o par de delimitadores é ``\verb/\left./'' e~``\verb/\right|/''.
O primeiro deles é invisível e serve apenas para delimitar a
expressão.\footnote{Precisamos informar ao computador qual é o início
  e o fim da expressão, para que ele possa calcular o quanto deve
  ampliar o delimitador.}

\chapter{Controle manual do tamanho}

Os diversos delimitadores ampliados podem ser acessados diretamente
usando os comandos~``\cs{big}'', ``\cs{Big}'', ``\cs{bigg}'',
e~``\cs{Bigg}''. Por exemplo, compare os resultados abaixo. À
esquerda, usamos dimensionamento automático (\cs{left} e \cs{right}),
e, à direita escolhemos manualmente o tamanho dos parênteses.
\[
\left(\sum_{a\in A} f(a) + b\right)^2 \quad\text{\emph{versus}}\quad \Bigl(\sum_{a\in A} f(a) + b \Bigl)^2.
\]

Abaixo estão alguns exemplos de delimitadores em todos esses tamanhos.

\newcommand{\ampe}{e} \newlength{\mylength}
\setlength{\mylength}{1\baselineskip}
\begin{exemplo}
\begin{tabular}{rc}
\astt{(} \ampe\ \astt{)} produzem 
  & \(\displaystyle \Bigg( \bigg( \Big( \big( ( a+\cdots+z ) \big) \Big) \bigg) \Bigg)\)\\[\mylength]
\astt{[} \ampe\ \astt{]} produzem 
  & \(\displaystyle \Bigg[ \bigg[ \Big[ \big[ [ a+\cdots+z ] \big] \Big] \bigg] \Bigg]\)\\[\mylength]
\astt{\{} \ampe\ \astt{\}} produzem
  & \(\displaystyle \Bigg\{ \bigg\{ \Big\{ \big\{ \{ a+\cdots+z \} \big\} \Big\} \bigg\} \Bigg\}\)\\[\mylength]
``\cs{langle}'' \ampe\ ``\cs{rangle}'' produzem
  & \(\displaystyle \Bigg\langle \bigg\langle \Big\langle \big\langle  \langle a+\cdots+z \rangle \big\rangle \Big\rangle \bigg\rangle \Bigg\rangle\)\\[\mylength]
``\cs{lceil}'' \ampe\ ``\cs{rceil}'' produzem
  & \(\displaystyle \Bigg\lceil \bigg\lceil \Big\lceil \big\lceil \lceil a+\cdots+z \rceil \big\rceil \Big\rceil \bigg\rceil \Bigg\rceil\)\\[\mylength]
``\cs{lceil}'' \ampe\ ``\cs{rceil}'' produzem
  & \(\displaystyle \Bigg\lfloor \bigg\lfloor \Big\lfloor \big\lfloor \lfloor a+\cdots+z \rfloor \big\rfloor \Big\rfloor \bigg\rfloor \Bigg\rfloor\)\\[\mylength]
\astt{|} produz
  & \(\displaystyle \Bigg| \bigg| \Big| \big| | a+\cdots+z | \big| \Big| \bigg| \Bigg|\)\\[\mylength]
\cs{|} produz
  & \(\displaystyle \Bigg\| \bigg\| \Big\| \big\| \| a+\cdots+z \| \big\| \Big\| \bigg\| \Bigg\|\)\\[\mylength]
\end{tabular}
\end{exemplo}
Mas a história não termina aqui. O \emph{espaçamento} entre
delimitadores e os elementos torno de si também pode ser
controlado. Cada um dos comandos aumentadores existe em três
modalidades: esquerdo (\emph{\textbf{L}eft}) , centro
(\emph{\textbf{M}iddle}) e direito (\emph{\textbf{R}ight}). Por
exemplo, \verb/\[\bigl\{ a + b \bigm| a\in A, b\in B \bigr\}\]/
resulta em
\[\bigl\{ a + b \bigm| a\in A, b\in B \bigr\},\]
ao passo \verb/\[\big\{ a + b \big| a\in A, b\in B \big\}\]/ resulta em
\[\big\{ a + b \big| a\in A, b\in B \big\},\]
e ainda \verb/\[\{ a + b | a\in A, b\in B \}\]/ resulta em
\[\{ a + b | a\in A, b\in B \}.\]
Compare ainda \verb/\[ x &= \sin\left(\frac12\right) \]/ com a (melhor) alternativa
\verb/\[ x = \sin\biggl(\frac12\biggr)\]/:
\begin{align*}
 x &= \sin\left(\frac12\right)  \\
 x &= \sin\biggl(\frac12\biggr)
\end{align*}


\chapter{Exercícios}


\end{document}
